% Options for packages loaded elsewhere
\PassOptionsToPackage{unicode}{hyperref}
\PassOptionsToPackage{hyphens}{url}
%
\documentclass[
]{book}
\usepackage{lmodern}
\usepackage{amssymb,amsmath}
\usepackage{ifxetex,ifluatex}
\ifnum 0\ifxetex 1\fi\ifluatex 1\fi=0 % if pdftex
  \usepackage[T1]{fontenc}
  \usepackage[utf8]{inputenc}
  \usepackage{textcomp} % provide euro and other symbols
\else % if luatex or xetex
  \usepackage{unicode-math}
  \defaultfontfeatures{Scale=MatchLowercase}
  \defaultfontfeatures[\rmfamily]{Ligatures=TeX,Scale=1}
\fi
% Use upquote if available, for straight quotes in verbatim environments
\IfFileExists{upquote.sty}{\usepackage{upquote}}{}
\IfFileExists{microtype.sty}{% use microtype if available
  \usepackage[]{microtype}
  \UseMicrotypeSet[protrusion]{basicmath} % disable protrusion for tt fonts
}{}
\makeatletter
\@ifundefined{KOMAClassName}{% if non-KOMA class
  \IfFileExists{parskip.sty}{%
    \usepackage{parskip}
  }{% else
    \setlength{\parindent}{0pt}
    \setlength{\parskip}{6pt plus 2pt minus 1pt}}
}{% if KOMA class
  \KOMAoptions{parskip=half}}
\makeatother
\usepackage{xcolor}
\IfFileExists{xurl.sty}{\usepackage{xurl}}{} % add URL line breaks if available
\IfFileExists{bookmark.sty}{\usepackage{bookmark}}{\usepackage{hyperref}}
\hypersetup{
  pdftitle={A Guide to Collection \& Calculation of Cadre Harmonise Food Security Indicators - beta version},
  pdfauthor={Research, Assesment and Monitoring (RAM) Team Regional Bureau Dakar World Food Programme rbd.vam@wfp.org},
  hidelinks,
  pdfcreator={LaTeX via pandoc}}
\urlstyle{same} % disable monospaced font for URLs
\usepackage{color}
\usepackage{fancyvrb}
\newcommand{\VerbBar}{|}
\newcommand{\VERB}{\Verb[commandchars=\\\{\}]}
\DefineVerbatimEnvironment{Highlighting}{Verbatim}{commandchars=\\\{\}}
% Add ',fontsize=\small' for more characters per line
\usepackage{framed}
\definecolor{shadecolor}{RGB}{248,248,248}
\newenvironment{Shaded}{\begin{snugshade}}{\end{snugshade}}
\newcommand{\AlertTok}[1]{\textcolor[rgb]{0.94,0.16,0.16}{#1}}
\newcommand{\AnnotationTok}[1]{\textcolor[rgb]{0.56,0.35,0.01}{\textbf{\textit{#1}}}}
\newcommand{\AttributeTok}[1]{\textcolor[rgb]{0.77,0.63,0.00}{#1}}
\newcommand{\BaseNTok}[1]{\textcolor[rgb]{0.00,0.00,0.81}{#1}}
\newcommand{\BuiltInTok}[1]{#1}
\newcommand{\CharTok}[1]{\textcolor[rgb]{0.31,0.60,0.02}{#1}}
\newcommand{\CommentTok}[1]{\textcolor[rgb]{0.56,0.35,0.01}{\textit{#1}}}
\newcommand{\CommentVarTok}[1]{\textcolor[rgb]{0.56,0.35,0.01}{\textbf{\textit{#1}}}}
\newcommand{\ConstantTok}[1]{\textcolor[rgb]{0.00,0.00,0.00}{#1}}
\newcommand{\ControlFlowTok}[1]{\textcolor[rgb]{0.13,0.29,0.53}{\textbf{#1}}}
\newcommand{\DataTypeTok}[1]{\textcolor[rgb]{0.13,0.29,0.53}{#1}}
\newcommand{\DecValTok}[1]{\textcolor[rgb]{0.00,0.00,0.81}{#1}}
\newcommand{\DocumentationTok}[1]{\textcolor[rgb]{0.56,0.35,0.01}{\textbf{\textit{#1}}}}
\newcommand{\ErrorTok}[1]{\textcolor[rgb]{0.64,0.00,0.00}{\textbf{#1}}}
\newcommand{\ExtensionTok}[1]{#1}
\newcommand{\FloatTok}[1]{\textcolor[rgb]{0.00,0.00,0.81}{#1}}
\newcommand{\FunctionTok}[1]{\textcolor[rgb]{0.00,0.00,0.00}{#1}}
\newcommand{\ImportTok}[1]{#1}
\newcommand{\InformationTok}[1]{\textcolor[rgb]{0.56,0.35,0.01}{\textbf{\textit{#1}}}}
\newcommand{\KeywordTok}[1]{\textcolor[rgb]{0.13,0.29,0.53}{\textbf{#1}}}
\newcommand{\NormalTok}[1]{#1}
\newcommand{\OperatorTok}[1]{\textcolor[rgb]{0.81,0.36,0.00}{\textbf{#1}}}
\newcommand{\OtherTok}[1]{\textcolor[rgb]{0.56,0.35,0.01}{#1}}
\newcommand{\PreprocessorTok}[1]{\textcolor[rgb]{0.56,0.35,0.01}{\textit{#1}}}
\newcommand{\RegionMarkerTok}[1]{#1}
\newcommand{\SpecialCharTok}[1]{\textcolor[rgb]{0.00,0.00,0.00}{#1}}
\newcommand{\SpecialStringTok}[1]{\textcolor[rgb]{0.31,0.60,0.02}{#1}}
\newcommand{\StringTok}[1]{\textcolor[rgb]{0.31,0.60,0.02}{#1}}
\newcommand{\VariableTok}[1]{\textcolor[rgb]{0.00,0.00,0.00}{#1}}
\newcommand{\VerbatimStringTok}[1]{\textcolor[rgb]{0.31,0.60,0.02}{#1}}
\newcommand{\WarningTok}[1]{\textcolor[rgb]{0.56,0.35,0.01}{\textbf{\textit{#1}}}}
\usepackage{longtable,booktabs}
% Correct order of tables after \paragraph or \subparagraph
\usepackage{etoolbox}
\makeatletter
\patchcmd\longtable{\par}{\if@noskipsec\mbox{}\fi\par}{}{}
\makeatother
% Allow footnotes in longtable head/foot
\IfFileExists{footnotehyper.sty}{\usepackage{footnotehyper}}{\usepackage{footnote}}
\makesavenoteenv{longtable}
\usepackage{graphicx,grffile}
\makeatletter
\def\maxwidth{\ifdim\Gin@nat@width>\linewidth\linewidth\else\Gin@nat@width\fi}
\def\maxheight{\ifdim\Gin@nat@height>\textheight\textheight\else\Gin@nat@height\fi}
\makeatother
% Scale images if necessary, so that they will not overflow the page
% margins by default, and it is still possible to overwrite the defaults
% using explicit options in \includegraphics[width, height, ...]{}
\setkeys{Gin}{width=\maxwidth,height=\maxheight,keepaspectratio}
% Set default figure placement to htbp
\makeatletter
\def\fps@figure{htbp}
\makeatother
\setlength{\emergencystretch}{3em} % prevent overfull lines
\providecommand{\tightlist}{%
  \setlength{\itemsep}{0pt}\setlength{\parskip}{0pt}}
\setcounter{secnumdepth}{5}
\usepackage{booktabs}
\usepackage{amsthm}
\makeatletter
\def\thm@space@setup{%
  \thm@preskip=8pt plus 2pt minus 4pt
  \thm@postskip=\thm@preskip
}
\makeatother
\usepackage[]{natbib}
\bibliographystyle{apalike}

\title{A Guide to Collection \& Calculation of Cadre Harmonise Food Security Indicators - beta version}
\author{Research, Assesment and Monitoring (RAM) Team Regional Bureau Dakar World Food Programme \href{mailto:rbd.vam@wfp.org}{\nolinkurl{rbd.vam@wfp.org}}}
\date{2021-02-12}

\begin{document}
\maketitle

{
\setcounter{tocdepth}{1}
\tableofcontents
}
\hypertarget{purpose-how-to-use-this-guide}{%
\chapter{Purpose \& How to use this Guide}\label{purpose-how-to-use-this-guide}}

This guide proposes a uniform way of collecting and analyzing key food security indicators for the Cadre Harmonise process.

Chapters are listed by indicator and include:

\begin{enumerate}
\def\labelenumi{\arabic{enumi}.}
\tightlist
\item
  standardized data collection modules and ODK compliant xlsforms
\item
  sample data set
\item
  standardized syntax in SPSS and R
\item
  references / official guidelines / source materials for the indicators
\end{enumerate}

\hypertarget{gecodes}{%
\chapter{Gecodes}\label{gecodes}}

Using names as identifiers can easily lead to confusion over spelling and transliterations as well as alternative and duplicate names.

As unique identifiers, P-Codes overcome the challenges of linking datasets and confirming locations.

We recommend using Gecodes and Names from OCHA's Common Operational Datasets:
\url{https://data.humdata.org/dashboards/cod?ext_geodata=1\&q=\&ext_page_size=25}

\hypertarget{a-brief-example-of-the-problem}{%
\section{A brief example of the problem:}\label{a-brief-example-of-the-problem}}

The geographic variables of the \emph{choices} section of the xlsform typically look like this

\begin{longtable}[]{@{}lll@{}}
\toprule
list\_name & name & label\tabularnewline
\midrule
\endhead
cod\_lga & 1 & ASKIRA/UBA\tabularnewline
cod\_lga & 2 & BAYO\tabularnewline
cod\_lga & 3 & CHIBOK\tabularnewline
cod\_lga & 4 & KALABAGE\tabularnewline
\bottomrule
\end{longtable}

The geographic variables in corresponding GIS dataset look like this:

\begin{longtable}[]{@{}lll@{}}
\toprule
Shape & admin2Pcode & admin2RefName\tabularnewline
\midrule
\endhead
Polygon & NG008002 & Askira/Uba\tabularnewline
Polygon & NG008004 & Bayo\tabularnewline
Polygon & NG008006 & Chibok\tabularnewline
Polygon & NG008015 & Kala/Balge\tabularnewline
\bottomrule
\end{longtable}

Unfortunately, if we tried to merge the dataset and the GIS file by the name of the area, we would first have to fix the capitalization and punctuation of the names.

To avoid unnecessary work, we recommend putting in the \emph{Pcode} in the name column of the xlsform:

\begin{longtable}[]{@{}lll@{}}
\toprule
list\_name & name & label\tabularnewline
\midrule
\endhead
cod\_lga & NG008002 & ASKIRA/UBA\tabularnewline
cod\_lga & NG008004 & BAYO\tabularnewline
cod\_lga & NG008006 & CHIBOK\tabularnewline
cod\_lga & NG008015 & KALABAGE\tabularnewline
\bottomrule
\end{longtable}

With a common, simple variable to merge both datasets, mapping and data processing will be much easier!

\hypertarget{where-to-find-pcodes}{%
\section{Where to find Pcodes}\label{where-to-find-pcodes}}

Geographic data can be found \url{https://data.humdata.org/dashboards/cod?ext_geodata=1\&q=\&ext_page_size=25} and searching for the appropriate country.

We recommend using the codes from the \emph{admin1Pcode} and \emph{admin2Pcode} columns in the \emph{name} column of the xlsform. Using the \emph{admin1RefName} and \emph{admin2RefName}

\hypertarget{example-data-set-of-the-data-collection-sheet-for-nigeria}{%
\section{Example data set of the data collection sheet for Nigeria:}\label{example-data-set-of-the-data-collection-sheet-for-nigeria}}

\href{https://github.com/WFP-VAM/RBD_FS_CH_guide_EN/blob/master/questionnaires/xlsformwithgeocodesexample.xlsx}{xlsformwithgeocodesexample}

\hypertarget{caveats}{%
\section{2 caveats}\label{caveats}}

\begin{enumerate}
\def\labelenumi{\arabic{enumi}.}
\item
  In some countries, the boundaries in the CODS might not correspond with the latest boundaries used by the country. In this case, special development of a GIS boundary file and codes will need to be developed.
\item
  Inserting the relevant pcodes in the \emph{name} column will probably require some \emph{vlookup} . This might be relatively straightforward when only replacing codes at the admin1 and admin2 level, however things might get more complicated and difficult if cascading columns and the \emph{choice\_filter} option is used at the adm3, adm4 level.
\end{enumerate}

\hypertarget{household-hunger-scale}{%
\chapter{Household Hunger Scale}\label{household-hunger-scale}}

\hypertarget{references-resources}{%
\section{References / Resources}\label{references-resources}}

\begin{enumerate}
\def\labelenumi{\arabic{enumi}.}
\tightlist
\item
  \url{https://www.fantaproject.org/monitoring-and-evaluation/household-hunger-scale-hhs}
\end{enumerate}

\hypertarget{standardized-questionnaire}{%
\section{Standardized Questionnaire}\label{standardized-questionnaire}}

\begin{longtable}[]{@{}lll@{}}
\toprule
\begin{minipage}[b]{0.17\columnwidth}\raggedright
Variable Name\strut
\end{minipage} & \begin{minipage}[b]{0.56\columnwidth}\raggedright
Question Label\strut
\end{minipage} & \begin{minipage}[b]{0.17\columnwidth}\raggedright
Answer Choices\strut
\end{minipage}\tabularnewline
\midrule
\endhead
\begin{minipage}[t]{0.17\columnwidth}\raggedright
HHhSNoFood\_YN\strut
\end{minipage} & \begin{minipage}[t]{0.56\columnwidth}\raggedright
In the past {[}4 weeks/30 days{]}, was there ever no food to eat of any kind in your house because of lack of resources to get food?\strut
\end{minipage} & \begin{minipage}[t]{0.17\columnwidth}\raggedright
1) Yes 0) No\strut
\end{minipage}\tabularnewline
\begin{minipage}[t]{0.17\columnwidth}\raggedright
HHhSNoFood\_FR\strut
\end{minipage} & \begin{minipage}[t]{0.56\columnwidth}\raggedright
How often did this happen in the past {[}4 weeks/30 days{]}?\strut
\end{minipage} & \begin{minipage}[t]{0.17\columnwidth}\raggedright
1) Rarely (1--2 times) 2) Sometimes (3--10 times) 3) Often (more than 10 times)\strut
\end{minipage}\tabularnewline
\begin{minipage}[t]{0.17\columnwidth}\raggedright
HHhSBedHung\_YN\strut
\end{minipage} & \begin{minipage}[t]{0.56\columnwidth}\raggedright
In the past {[}4 weeks/30 days{]}, did you or any household member go to sleep at night hungry because there was not enough food?\strut
\end{minipage} & \begin{minipage}[t]{0.17\columnwidth}\raggedright
1) Yes 0) No\strut
\end{minipage}\tabularnewline
\begin{minipage}[t]{0.17\columnwidth}\raggedright
HHhSBedHung\_FR\strut
\end{minipage} & \begin{minipage}[t]{0.56\columnwidth}\raggedright
How often did this happen in the past {[}4 weeks/30 days{]}?\strut
\end{minipage} & \begin{minipage}[t]{0.17\columnwidth}\raggedright
1) Rarely (1--2 times) 2) Sometimes (3--10 times) 3) Often (more than 10 times)\strut
\end{minipage}\tabularnewline
\begin{minipage}[t]{0.17\columnwidth}\raggedright
HHhSNotEat\_YN\strut
\end{minipage} & \begin{minipage}[t]{0.56\columnwidth}\raggedright
In the past {[}4 weeks/30 days{]}, did you or any household member go to sleep at night hungry because there was not enough food?\strut
\end{minipage} & \begin{minipage}[t]{0.17\columnwidth}\raggedright
1) Yes 0) No\strut
\end{minipage}\tabularnewline
\begin{minipage}[t]{0.17\columnwidth}\raggedright
HHhSNotEat\_FR\strut
\end{minipage} & \begin{minipage}[t]{0.56\columnwidth}\raggedright
How often did this happen in the past {[}4 weeks/30 days{]}?\strut
\end{minipage} & \begin{minipage}[t]{0.17\columnwidth}\raggedright
1) Rarely (1--2 times) 2) Sometimes (3--10 times) 3) Often (more than 10 times)\strut
\end{minipage}\tabularnewline
\bottomrule
\end{longtable}

\hypertarget{paper-version-of-questionnaire}{%
\subsection{Paper Version of Questionnaire}\label{paper-version-of-questionnaire}}

Here is the standardized module in word format:
\href{https://github.com/WFP-VAM/RBD_FS_CH_guide_EN/blob/master/questionnaires/RBDstandardized_questionnaireHHS.xlsx}{RBDstandardized\_questionnaireHHS}

\hypertarget{electronic-version-of-questionnaire}{%
\subsection{Electronic Version of Questionnaire}\label{electronic-version-of-questionnaire}}

Here is the standardized module in xlsform format:
\href{https://github.com/WFP-VAM/RBD_FS_CH_guide_EN/blob/master/questionnaires/RBDstandardized_questionnaireHHS.docx}{RBDstandardized\_questionnaireHHS}

\hypertarget{calculation-of-household-hunger-scale-indicators-cadre-harmonise-phasing}{%
\section{Calculation of Household Hunger Scale Indicators \& Cadre Harmonise Phasing}\label{calculation-of-household-hunger-scale-indicators-cadre-harmonise-phasing}}

\hypertarget{example-data-set}{%
\subsection{Example data set}\label{example-data-set}}

Here is the example data set:
\href{https://github.com/WFP-VAM/RBD_FS_CH_guide_EN/blob/master/example_datasets/dataHHSEng.sav}{dataHHSEng}

\hypertarget{spss-syntax}{%
\subsection{SPSS Syntax}\label{spss-syntax}}

\begin{Shaded}
\begin{Highlighting}[]
\NormalTok{GET  FILE=}\StringTok{'dataHHSEng.sav'}\NormalTok{.}
\NormalTok{DATASET NAME DataSet1 WINDOW=FRONT.}

\OperatorTok{**}\StringTok{ }\NormalTok{recode the frequency questions to scores}

\NormalTok{Recode HHhSNoFood_FR HHhSBedHung_FR }\KeywordTok{HHhSNotEat_FR}\NormalTok{ (}\DecValTok{1}\NormalTok{ =}\StringTok{ }\DecValTok{1}\NormalTok{) (}\DecValTok{2}\NormalTok{=}\DecValTok{1}\NormalTok{) (}\DecValTok{3}\NormalTok{=}\DecValTok{2}\NormalTok{) (}\DataTypeTok{ELSE=}\DecValTok{0}\NormalTok{) INTO HHhSNoFood_FR_r HHhSBedHung_FR_r HHhSNotEat_FR_r.}

\NormalTok{Variable labels HHhSNoFood_FR_r }\StringTok{"In the past [4 weeks/30 days], was there ever no food to eat of any kind in your house because of lack of resources to get food? - recoded"}
\NormalTok{HHhSBedHung_FR_r }\StringTok{"In the past [4 weeks/30 days], did you or any household member go to sleep at night hungry because there was not enough food? - recoded"}
\NormalTok{HHhSNotEat_FR_r }\StringTok{"In the past [4 weeks/30 days], did you or any household member go to sleep at night hungry because there was not enough food? - recoded"}\NormalTok{.}

\OperatorTok{**}\StringTok{ }\NormalTok{sum the recoded questions to calculate the HHS}

\NormalTok{Compute HHhS =}\StringTok{ }\NormalTok{HHhSNoFood_FR_r }\OperatorTok{+}\StringTok{ }\NormalTok{HHhSBedHung_FR_r }\OperatorTok{+}\StringTok{ }\NormalTok{HHhSNotEat_FR_r.}
\NormalTok{variable labels HHhS }\StringTok{"Household Hunger Scale"}\NormalTok{.}

\OperatorTok{**}\StringTok{ }\NormalTok{each household should have an HHhS score between }\DecValTok{0} \OperatorTok{-}\StringTok{ }\FloatTok{6.}

\NormalTok{FREQUENCIES VARIABLES =}\StringTok{ }\NormalTok{HHhS}
\OperatorTok{/}\NormalTok{STATISTICS=MEAN MEDIAN MINIMUM MAXIMUM}
\OperatorTok{/}\NormalTok{ORDER=ANALYSIS.}

\OperatorTok{**}\StringTok{ }\NormalTok{Create Categorical HHhS}

\NormalTok{RECODE }\KeywordTok{HHhS}\NormalTok{ (}\DecValTok{0}\NormalTok{ thru }\DecValTok{1}\NormalTok{=}\DecValTok{1}\NormalTok{) (}\DecValTok{2}\NormalTok{ thru }\DecValTok{3}\NormalTok{=}\DecValTok{2}\NormalTok{) (}\DecValTok{4}\NormalTok{ thru }\DataTypeTok{Highest=}\DecValTok{3}\NormalTok{) INTO HHhSCat.}
\NormalTok{variable labels HHhSCat }\StringTok{"Household Hunger Score Categories"}\NormalTok{.}
\NormalTok{value labels HHhSCat }
\DecValTok{1} \StringTok{`}\DataTypeTok{No or little hunger in the household}\StringTok{`}
\DecValTok{2} \StringTok{`}\DataTypeTok{Moderate hunger in the household}\StringTok{`}
\DecValTok{3} \StringTok{`}\DataTypeTok{Severe hunger in the household}\StringTok{`}\NormalTok{.}

\OperatorTok{**}\StringTok{ }\NormalTok{Create HHhS Cadre Harmonise Categories}

\NormalTok{RECODE }\KeywordTok{HHS}\NormalTok{ (}\DecValTok{0}\NormalTok{=}\DecValTok{1}\NormalTok{) (}\DecValTok{1}\NormalTok{=}\DecValTok{2}\NormalTok{) (}\DecValTok{2}\NormalTok{ thru }\DecValTok{3}\NormalTok{=}\DecValTok{3}\NormalTok{) (}\DecValTok{4}\NormalTok{=}\DecValTok{4}\NormalTok{) (}\DecValTok{5}\NormalTok{=}\DecValTok{5}\NormalTok{) INTO HHhS_CH.}
\NormalTok{variable labels HHhS_CH }\StringTok{"Household Hunger Score Categories - Cadre Harmonise"}\NormalTok{.}
\NormalTok{value labels HHhS_CH}
\DecValTok{1} \StringTok{`}\DataTypeTok{Phase1}\StringTok{`}
\DecValTok{2} \StringTok{`}\DataTypeTok{Phase2}\StringTok{`}
\DecValTok{3} \StringTok{`}\DataTypeTok{Phase3}\StringTok{`}
\DecValTok{4} \StringTok{`}\DataTypeTok{Phase4}\StringTok{`}
\DecValTok{5} \StringTok{`}\DataTypeTok{Phase5}\StringTok{`}\NormalTok{.}

\OperatorTok{**}\StringTok{ }\NormalTok{Generate table of proportion of households }\ControlFlowTok{in}\NormalTok{ CH HHS phases by Adm1 and Adm2 using weights}

\NormalTok{WEIGHT BY WeightHH.}

\NormalTok{CROSSTABS}
  \OperatorTok{/}\NormalTok{TABLES=ADMIN2Name BY HHhS_CH BY ADMIN1Name}
  \OperatorTok{/}\NormalTok{FORMAT=AVALUE TABLES}
  \OperatorTok{/}\NormalTok{CELLS=ROW }
  \OperatorTok{/}\NormalTok{COUNT ROUND CELL.}
\end{Highlighting}
\end{Shaded}

Here is the SPSS syntax file:

\href{https://github.com/WFP-VAM/RBD_FS_CH_guide_EN/blob/master/syntax/RBDstandardized_SPSSsyntaxHHS.sps}{RBDstandardized\_SPSSsyntaxHHS}

\hypertarget{r-syntax}{%
\subsection{R Syntax}\label{r-syntax}}

\begin{Shaded}
\begin{Highlighting}[]
\KeywordTok{library}\NormalTok{(haven)}
\KeywordTok{library}\NormalTok{(labelled)}
\KeywordTok{library}\NormalTok{(tidyverse)}

\CommentTok{#import dataset}
\NormalTok{dataHHSEng <-}\StringTok{ }\KeywordTok{read_sav}\NormalTok{(}\StringTok{"example_datasets/dataHHSEng.sav"}\NormalTok{)}

\CommentTok{#Calculate HHS }
\NormalTok{dataHHSEng <-}\StringTok{ }\KeywordTok{to_factor}\NormalTok{(dataHHSEng)}

\CommentTok{#Recode HHS questions into new variables with score }
\NormalTok{dataHHSEng <-}\StringTok{ }\NormalTok{dataHHSEng }\OperatorTok\StringTok{ }\KeywordTok{mutate}\NormalTok{(}\DataTypeTok{HHhSNoFood_FR_r =} \KeywordTok{case_when}\NormalTok{(}
\NormalTok{                                      HHhSNoFood_FR }\OperatorTok{==}\StringTok{ "Rarely (1–2 times)"} \OperatorTok{~}\StringTok{ }\DecValTok{1}\NormalTok{,}
\NormalTok{                                      HHhSNoFood_FR }\OperatorTok{==}\StringTok{ "Sometimes (3–10 times)"} \OperatorTok{~}\StringTok{ }\DecValTok{1}\NormalTok{,}
\NormalTok{                                      HHhSNoFood_FR }\OperatorTok{==}\StringTok{ "Often (more than 10 times)"} \OperatorTok{~}\StringTok{ }\DecValTok{2}\NormalTok{,}
                                      \OtherTok{TRUE} \OperatorTok{~}\StringTok{ }\DecValTok{0}\NormalTok{),}
                                    \DataTypeTok{HHhSBedHung_FR_r =} \KeywordTok{case_when}\NormalTok{(}
\NormalTok{                                      HHhSBedHung_FR }\OperatorTok{==}\StringTok{ "Rarely (1–2 times)"} \OperatorTok{~}\StringTok{ }\DecValTok{1}\NormalTok{,}
\NormalTok{                                      HHhSBedHung_FR }\OperatorTok{==}\StringTok{ "Sometimes (3–10 times)"} \OperatorTok{~}\StringTok{ }\DecValTok{1}\NormalTok{,}
\NormalTok{                                      HHhSBedHung_FR }\OperatorTok{==}\StringTok{ "Often (more than 10 times)"} \OperatorTok{~}\StringTok{ }\DecValTok{2}\NormalTok{,}
                                      \OtherTok{TRUE} \OperatorTok{~}\StringTok{ }\DecValTok{0}\NormalTok{),}
                                    \DataTypeTok{HHhSNotEat_FR_r =} \KeywordTok{case_when}\NormalTok{(}
\NormalTok{                                      HHhSNotEat_FR }\OperatorTok{==}\StringTok{ "Rarely (1–2 times)"} \OperatorTok{~}\StringTok{ }\DecValTok{1}\NormalTok{,}
\NormalTok{                                      HHhSNotEat_FR }\OperatorTok{==}\StringTok{ "Sometimes (3–10 times)"} \OperatorTok{~}\StringTok{ }\DecValTok{1}\NormalTok{,}
\NormalTok{                                      HHhSNotEat_FR }\OperatorTok{==}\StringTok{ "Often (more than 10 times)"} \OperatorTok{~}\StringTok{ }\DecValTok{2}\NormalTok{,}
                                      \OtherTok{TRUE} \OperatorTok{~}\StringTok{ }\DecValTok{0}\NormalTok{))}
\CommentTok{# Calculate HHhS score}
\NormalTok{dataHHSEng <-}\StringTok{ }\NormalTok{dataHHSEng }\OperatorTok\StringTok{ }\KeywordTok{mutate}\NormalTok{(}\DataTypeTok{HHhS =}\NormalTok{ HHhSNoFood_FR_r }\OperatorTok{+}\StringTok{ }\NormalTok{HHhSBedHung_FR_r }\OperatorTok{+}\StringTok{ }\NormalTok{HHhSNotEat_FR_r)}
\KeywordTok{var_label}\NormalTok{(dataHHSEng}\OperatorTok{$}\NormalTok{HHhS) <-}\StringTok{ "Household Hunger Scale"}

\CommentTok{#each household should have an HHS score between 0 - 6}
\KeywordTok{summary}\NormalTok{(dataHHSEng}\OperatorTok{$}\NormalTok{HHhS)}

\CommentTok{# Create Categorical HHS}
\NormalTok{dataHHSEng <-}\StringTok{ }\NormalTok{dataHHSEng }\OperatorTok\StringTok{ }\KeywordTok{mutate}\NormalTok{(}\DataTypeTok{HHhSCat =} \KeywordTok{case_when}\NormalTok{(}
\NormalTok{    HHhS }\OperatorTok\StringTok{ }\KeywordTok{c}\NormalTok{(}\DecValTok{0}\NormalTok{,}\DecValTok{1}\NormalTok{) }\OperatorTok{~}\StringTok{ "No or little hunger in the household"}\NormalTok{,}
\NormalTok{    HHhS }\OperatorTok\StringTok{ }\KeywordTok{c}\NormalTok{(}\DecValTok{2}\NormalTok{,}\DecValTok{3}\NormalTok{) }\OperatorTok{~}\StringTok{ "Moderate hunger in the household"}\NormalTok{,}
\NormalTok{    HHhS }\OperatorTok{>=}\StringTok{ }\DecValTok{4} \OperatorTok{~}\StringTok{ "Severe hunger in the household"} 
\NormalTok{    ))}
\KeywordTok{var_label}\NormalTok{(dataHHSEng}\OperatorTok{$}\NormalTok{HHhSCat) <-}\StringTok{ "Household Hunger Score Categories"}

\CommentTok{#Convert HH Scores to CH phases}
\NormalTok{dataHHSEng <-}\StringTok{ }\NormalTok{dataHHSEng }\OperatorTok\StringTok{ }\KeywordTok{mutate}\NormalTok{(}\DataTypeTok{HHhS_CH =} \KeywordTok{case_when}\NormalTok{(}
\NormalTok{    HHhS }\OperatorTok{==}\StringTok{ }\DecValTok{0} \OperatorTok{~}\StringTok{ "Phase1"}\NormalTok{,}
\NormalTok{    HHhS }\OperatorTok{==}\StringTok{ }\DecValTok{1} \OperatorTok{~}\StringTok{ "Phase2"}\NormalTok{,}
\NormalTok{    HHhS }\OperatorTok\StringTok{ }\KeywordTok{c}\NormalTok{(}\DecValTok{2}\NormalTok{,}\DecValTok{3}\NormalTok{) }\OperatorTok{~}\StringTok{ "Phase3"}\NormalTok{,}
\NormalTok{    HHhS }\OperatorTok{==}\StringTok{ }\DecValTok{4} \OperatorTok{~}\StringTok{ "Phase4"}\NormalTok{,  }
\NormalTok{    HHhS }\OperatorTok{>=}\StringTok{ }\DecValTok{5} \OperatorTok{~}\StringTok{ "Phase5"}\NormalTok{))}
\KeywordTok{var_label}\NormalTok{(dataHHSEng}\OperatorTok{$}\NormalTok{HHhS_CH) <-}\StringTok{ "Household Hunger Score Categories - Cadre Harmonise"}

\CommentTok{#Generate table of proportion of households in CH HHS phases by Adm1 and Adm2 using weights}
\NormalTok{CH_HHhS_table_wide <-}\StringTok{ }\NormalTok{dataHHSEng }\OperatorTok\StringTok{ }\KeywordTok{group_by}\NormalTok{(ADMIN1Name, ADMIN2Name, ADMIN2Code) }\OperatorTok
\StringTok{  }\KeywordTok{drop_na}\NormalTok{(HHhS_CH) }\OperatorTok
\StringTok{  }\KeywordTok{count}\NormalTok{(HHhS_CH, }\DataTypeTok{wt =}\NormalTok{ WeightHH) }\OperatorTok
\StringTok{  }\KeywordTok{mutate}\NormalTok{(}\DataTypeTok{perc =} \DecValTok{100} \OperatorTok{*}\StringTok{ }\NormalTok{n }\OperatorTok{/}\StringTok{ }\KeywordTok{sum}\NormalTok{(n)) }\OperatorTok
\StringTok{  }\KeywordTok{ungroup}\NormalTok{() }\OperatorTok\StringTok{ }\KeywordTok{select}\NormalTok{(}\OperatorTok{-}\NormalTok{n) }\OperatorTok
\StringTok{  }\KeywordTok{spread}\NormalTok{(}\DataTypeTok{key =}\NormalTok{ HHhS_CH, }\DataTypeTok{value =}\NormalTok{ perc) }\OperatorTok\StringTok{ }\KeywordTok{replace}\NormalTok{(., }\KeywordTok{is.na}\NormalTok{(.), }\DecValTok{0}\NormalTok{)  }\OperatorTok\StringTok{ }\KeywordTok{mutate_if}\NormalTok{(is.numeric, round, }\DecValTok{1}\NormalTok{)}

\CommentTok{#Calculate phasing of CH HHS indicator for area (applying CH 20% rules)}
\NormalTok{CH_HHhS_table_wide <-}\StringTok{ }\NormalTok{CH_HHhS_table_wide }\OperatorTok\StringTok{ }\KeywordTok{mutate}\NormalTok{(}\DataTypeTok{phase2345 =} \StringTok{`}\DataTypeTok{Phase2}\StringTok{`} \OperatorTok{+}\StringTok{ `}\DataTypeTok{Phase3}\StringTok{`} \OperatorTok{+}\StringTok{ `}\DataTypeTok{Phase4}\StringTok{`} \OperatorTok{+}\StringTok{ `}\DataTypeTok{Phase5}\StringTok{`}\NormalTok{,}\DataTypeTok{phase345 =} \StringTok{`}\DataTypeTok{Phase3}\StringTok{`} \OperatorTok{+}\StringTok{ `}\DataTypeTok{Phase4}\StringTok{`} \OperatorTok{+}\StringTok{ `}\DataTypeTok{Phase5}\StringTok{`}\NormalTok{, }\DataTypeTok{phase45 =} \StringTok{`}\DataTypeTok{Phase4}\StringTok{`} \OperatorTok{+}\StringTok{ `}\DataTypeTok{Phase5}\StringTok{`}\NormalTok{,}
      \DataTypeTok{HHS_finalphase =} \KeywordTok{case_when}\NormalTok{(}
\NormalTok{      Phase5 }\OperatorTok{>=}\StringTok{ }\DecValTok{20} \OperatorTok{~}\StringTok{ }\DecValTok{5}\NormalTok{,}
\NormalTok{      Phase4 }\OperatorTok{>=}\StringTok{ }\DecValTok{20} \OperatorTok{|}\StringTok{ }\NormalTok{phase45 }\OperatorTok{>=}\StringTok{ }\DecValTok{20} \OperatorTok{~}\StringTok{ }\DecValTok{4}\NormalTok{,}
\NormalTok{      Phase3 }\OperatorTok{>=}\StringTok{ }\DecValTok{20} \OperatorTok{|}\StringTok{ }\NormalTok{phase345 }\OperatorTok{>=}\StringTok{ }\DecValTok{20} \OperatorTok{~}\StringTok{ }\DecValTok{3}\NormalTok{,}
\NormalTok{      Phase2 }\OperatorTok{>=}\StringTok{ }\DecValTok{20} \OperatorTok{|}\StringTok{ }\NormalTok{phase2345 }\OperatorTok{>=}\StringTok{ }\DecValTok{20} \OperatorTok{~}\StringTok{ }\DecValTok{2}\NormalTok{,}
      \OtherTok{TRUE} \OperatorTok{~}\StringTok{ }\DecValTok{1}\NormalTok{)) }\OperatorTok\StringTok{ }
\StringTok{  }\KeywordTok{select}\NormalTok{(ADMIN1Name, ADMIN2Name, ADMIN2Code, }\DataTypeTok{HHS_Phase1 =}\NormalTok{ Phase1, }\DataTypeTok{HHS_Phase2 =}\NormalTok{ Phase2, }\DataTypeTok{HHS_Phase3 =}\NormalTok{ Phase3, }\DataTypeTok{HHS_Phase4 =}\NormalTok{ Phase4, }\DataTypeTok{HHS_Phase5 =}\NormalTok{ Phase5, HHS_finalphase)}
\end{Highlighting}
\end{Shaded}

Here is the R syntax file:

\href{https://github.com/WFP-VAM/RBD_FS_CH_guide_EN/blob/master/syntax/RBDstandardized_RsyntaxHHS.R}{RBDstandardized\_RsyntaxHHS}

\hypertarget{reduced-coping-strategy-index}{%
\chapter{reduced Coping Strategy Index}\label{reduced-coping-strategy-index}}

\hypertarget{references-resources-1}{%
\section{References / Resources}\label{references-resources-1}}

\begin{enumerate}
\def\labelenumi{\arabic{enumi}.}
\item
  \url{https://documents.wfp.org/stellent/groups/public/documents/manual_guide_proced/wfp211058.pdf}
\item
  \url{https://documents.wfp.org/stellent/groups/public/documents/manual_guide_proced/wfp271449.pdf}
\end{enumerate}

\hypertarget{standardized-questionnaire-1}{%
\section{Standardized Questionnaire}\label{standardized-questionnaire-1}}

\begin{longtable}[]{@{}ll@{}}
\toprule
\begin{minipage}[b]{0.22\columnwidth}\raggedright
Variable Name\strut
\end{minipage} & \begin{minipage}[b]{0.72\columnwidth}\raggedright
Question Label\strut
\end{minipage}\tabularnewline
\midrule
\endhead
\begin{minipage}[t]{0.22\columnwidth}\raggedright
rCSILessQlty\strut
\end{minipage} & \begin{minipage}[t]{0.72\columnwidth}\raggedright
In the past 7 days, how many days has your household had to: \emph{Rely on less preferred and less expensive food} because you did not have enough food or money to buy food?\strut
\end{minipage}\tabularnewline
\begin{minipage}[t]{0.22\columnwidth}\raggedright
rCSIBorrow\strut
\end{minipage} & \begin{minipage}[t]{0.72\columnwidth}\raggedright
In the past 7 days, how many days has your household had to: \emph{Borrow food or rely on help from a relative or friend} because you did not have enough food or money to buy food?\strut
\end{minipage}\tabularnewline
\begin{minipage}[t]{0.22\columnwidth}\raggedright
rCSIMealSize\strut
\end{minipage} & \begin{minipage}[t]{0.72\columnwidth}\raggedright
In the past 7 days, how many days has your household had to: \emph{Limit portion size of meals} because you did not have enough food or money to buy food?\strut
\end{minipage}\tabularnewline
\begin{minipage}[t]{0.22\columnwidth}\raggedright
rCSIMealAdult\strut
\end{minipage} & \begin{minipage}[t]{0.72\columnwidth}\raggedright
In the past 7 days, how many days has your household had to: \emph{Restrict consumption by adults in order for small children to eat} because you did not have enough food or money to buy food?\strut
\end{minipage}\tabularnewline
\begin{minipage}[t]{0.22\columnwidth}\raggedright
rCSIMealNb\strut
\end{minipage} & \begin{minipage}[t]{0.72\columnwidth}\raggedright
In the past 7 days, how many days has your household had to: \emph{Reduce the number of meals eaten per day} because you did not have enough food or money to buy food?\strut
\end{minipage}\tabularnewline
\bottomrule
\end{longtable}

\hypertarget{paper-version-of-questionnaire-1}{%
\subsection{Paper Version of Questionnaire}\label{paper-version-of-questionnaire-1}}

Here is the standardized module in word format:
\href{https://github.com/WFP-VAM/RBD_FS_CH_guide_EN/blob/master/questionnaires/RBDstandardized_questionnairerCSI.docx}{RBDstandardized\_questionnairerCSI}

\hypertarget{electronic-version-of-questionnaire-1}{%
\subsection{Electronic Version of Questionnaire}\label{electronic-version-of-questionnaire-1}}

Here is the standardized module in xlsform:
\href{https://github.com/WFP-VAM/RBD_FS_CH_guide_EN/blob/master/questionnaires/RBDstandardized_questionnairerCSI.xlsx}{RBDstandardized\_questionnairerCSI}

\hypertarget{calculation-of-rcsi-indicator-ch}{%
\section{Calculation of rCSI Indicator \& CH}\label{calculation-of-rcsi-indicator-ch}}

\hypertarget{example-data-set-1}{%
\subsection{Example data set}\label{example-data-set-1}}

Here is the example data set:
\href{https://github.com/WFP-VAM/RBD_FS_CH_guide_EN/blob/master/example_datasets/datarCSIEng.sav}{datarCSIEng}

\hypertarget{spss-syntax-1}{%
\subsection{SPSS Syntax}\label{spss-syntax-1}}

\begin{Shaded}
\begin{Highlighting}[]
\NormalTok{GET  FILE=}\StringTok{'datarCSIEng.sav'}\NormalTok{.}
\NormalTok{DATASET NAME DataSet1 WINDOW=FRONT.}

\OperatorTok{**}\StringTok{ }\NormalTok{caculate rCSI}

\NormalTok{compute rCSI =}\StringTok{ }\KeywordTok{sum}\NormalTok{(rCSILessQlty,rCSIBorrow}\OperatorTok{*}\DecValTok{2}\NormalTok{,rCSIMealSize,rCSIMealAdult}\OperatorTok{*}\DecValTok{3}\NormalTok{,rCSIMealNb).}
\NormalTok{Variable labels rCSI }\StringTok{"rCSI"}\NormalTok{.}

\OperatorTok{**}\StringTok{ }\NormalTok{each household should have a rCSI between }\DecValTok{0} \OperatorTok{-}\StringTok{ }\DecValTok{56}

\NormalTok{FREQUENCIES VARIABLES =}\StringTok{  }\NormalTok{rCSI}
\OperatorTok{/}\NormalTok{STATISTICS=MEAN MEDIAN MINIMUM MAXIMUM}
\OperatorTok{/}\NormalTok{ORDER=ANALYSIS.}

\OperatorTok{**}\StringTok{ }\NormalTok{Create rCSI Cadre Harmonise Categories}

\NormalTok{RECODE }\KeywordTok{rCSI}\NormalTok{ (}\DecValTok{0}\NormalTok{ thru }\DecValTok{3}\NormalTok{=}\DecValTok{1}\NormalTok{) (}\DecValTok{4}\NormalTok{ thru }\DecValTok{18}\NormalTok{=}\DecValTok{2}\NormalTok{) (}\DecValTok{19}\NormalTok{ thru }\DecValTok{56}\NormalTok{=}\DecValTok{3}\NormalTok{) INTO rCSI_CH.}
\NormalTok{variable labels rCSI_CH }\StringTok{"rCSI categories - Cadre Harmonise "}\NormalTok{.}
\NormalTok{value labels rCSI_CH}
\DecValTok{1} \StringTok{`}\DataTypeTok{Phase1}\StringTok{`}
\DecValTok{2} \StringTok{`}\DataTypeTok{Phase2}\StringTok{`}
\DecValTok{3} \StringTok{`}\DataTypeTok{Phase3}\StringTok{`}\NormalTok{.}

\OperatorTok{**}\StringTok{ }\NormalTok{Generate table of proportion of households }\ControlFlowTok{in}\NormalTok{ CH HHhS phases by Adm1 and Adm2 using weights}

\NormalTok{WEIGHT BY WeightHH.}

\NormalTok{CROSSTABS}
  \OperatorTok{/}\NormalTok{TABLES=ADMIN2Name BY rCSI_CH BY ADMIN1Name}
  \OperatorTok{/}\NormalTok{FORMAT=AVALUE TABLES}
  \OperatorTok{/}\NormalTok{CELLS=ROW }
  \OperatorTok{/}\NormalTok{COUNT ROUND CELL.}
\end{Highlighting}
\end{Shaded}

Here is the SPSS syntax file:

\href{https://github.com/WFP-VAM/RBD_FS_CH_guide_EN/blob/master/syntax/RBDstandardized_SPSSsyntaxrCSI.sps}{RBDstandardized\_SPSSsyntaxrCSI}

\hypertarget{r-syntax-1}{%
\subsection{R Syntax}\label{r-syntax-1}}

\begin{Shaded}
\begin{Highlighting}[]
\KeywordTok{library}\NormalTok{(haven)}
\KeywordTok{library}\NormalTok{(labelled)}
\KeywordTok{library}\NormalTok{(tidyverse)}

\CommentTok{#import dataset}
\NormalTok{datarCSIEng <-}\StringTok{ }\KeywordTok{read_sav}\NormalTok{(}\StringTok{"example_datasets/datarCSIEng.sav"}\NormalTok{)}

\CommentTok{#Calculate HHS }
\NormalTok{datarCSIEng <-}\StringTok{ }\KeywordTok{to_factor}\NormalTok{(datarCSIEng)}

\CommentTok{#calculate rCSI Score}
\NormalTok{datarCSIEng <-}\StringTok{ }\NormalTok{datarCSIEng }\OperatorTok\StringTok{ }\KeywordTok{mutate}\NormalTok{(}\DataTypeTok{rCSI =}\NormalTok{ rCSILessQlty  }\OperatorTok{+}\StringTok{ }\NormalTok{(}\DecValTok{2} \OperatorTok{*}\StringTok{ }\NormalTok{rCSIBorrow) }\OperatorTok{+}\StringTok{ }\NormalTok{rCSIMealSize }\OperatorTok{+}\StringTok{ }\NormalTok{(}\DecValTok{3} \OperatorTok{*}\StringTok{ }\NormalTok{rCSIMealAdult) }\OperatorTok{+}\StringTok{ }\NormalTok{rCSIMealNb)}
\KeywordTok{var_label}\NormalTok{(datarCSIEng}\OperatorTok{$}\NormalTok{rCSI) <-}\StringTok{ "rCSI"}

\CommentTok{#each household should have a rCSI between 0 - 56}

\KeywordTok{summary}\NormalTok{(datarCSIEng}\OperatorTok{$}\NormalTok{rCSI)}

\CommentTok{#Create rCSI Cadre Harmonise Categories}
\NormalTok{datarCSIEng <-}\StringTok{ }\NormalTok{datarCSIEng }\OperatorTok\StringTok{ }\KeywordTok{mutate}\NormalTok{(}\DataTypeTok{rCSI_CH =} \KeywordTok{case_when}\NormalTok{(}
\NormalTok{  rCSI }\OperatorTok{<=}\StringTok{ }\DecValTok{3} \OperatorTok{~}\StringTok{ "Phase1"}\NormalTok{, }
  \KeywordTok{between}\NormalTok{(rCSI,}\DecValTok{4}\NormalTok{,}\DecValTok{18}\NormalTok{) }\OperatorTok{~}\StringTok{ "Phase2"}\NormalTok{,       }
\NormalTok{  rCSI }\OperatorTok{>=}\StringTok{ }\DecValTok{19} \OperatorTok{~}\StringTok{ "Phase3"}\NormalTok{))}
\KeywordTok{var_label}\NormalTok{(datarCSIEng}\OperatorTok{$}\NormalTok{rCSI_CH) <-}\StringTok{ "rCSI categories - Cadre Harmonise "}

\CommentTok{#Generate table of proportion of households in CH rCSI phases by Adm1 and Adm2 using weights}
\NormalTok{CH_rCSI_table_wide <-}\StringTok{ }\NormalTok{datarCSIEng }\OperatorTok\StringTok{ }\KeywordTok{group_by}\NormalTok{(ADMIN1Name, ADMIN2Name, ADMIN2Code) }\OperatorTok
\StringTok{  }\KeywordTok{drop_na}\NormalTok{(rCSI_CH) }\OperatorTok
\StringTok{  }\KeywordTok{count}\NormalTok{(rCSI_CH, }\DataTypeTok{wt =}\NormalTok{ WeightHH) }\OperatorTok
\StringTok{  }\KeywordTok{mutate}\NormalTok{(}\DataTypeTok{perc =} \DecValTok{100} \OperatorTok{*}\StringTok{ }\NormalTok{n }\OperatorTok{/}\StringTok{ }\KeywordTok{sum}\NormalTok{(n)) }\OperatorTok
\StringTok{  }\KeywordTok{ungroup}\NormalTok{() }\OperatorTok\StringTok{ }\KeywordTok{select}\NormalTok{(}\OperatorTok{-}\NormalTok{n) }\OperatorTok
\StringTok{  }\KeywordTok{spread}\NormalTok{(}\DataTypeTok{key =}\NormalTok{ rCSI_CH, }\DataTypeTok{value =}\NormalTok{ perc) }\OperatorTok\StringTok{ }\KeywordTok{replace}\NormalTok{(., }\KeywordTok{is.na}\NormalTok{(.), }\DecValTok{0}\NormalTok{)  }\OperatorTok\StringTok{ }\KeywordTok{mutate_if}\NormalTok{(is.numeric, round, }\DecValTok{1}\NormalTok{)}


\CommentTok{#Calculate phasing of CH rCSI indicator for area (applying CH 20% rules)}
\NormalTok{CH_rCSI_table_wide <-}\StringTok{ }\NormalTok{CH_rCSI_table_wide  }\OperatorTok\StringTok{  }\KeywordTok{mutate}\NormalTok{(}\DataTypeTok{rcsi23 =}\NormalTok{ Phase2 }\OperatorTok{+}\StringTok{ }\NormalTok{Phase3,}
         \DataTypeTok{rCSI_finalphase =}
           \KeywordTok{case_when}\NormalTok{(}
\NormalTok{             Phase3 }\OperatorTok{>=}\StringTok{ }\DecValTok{20} \OperatorTok{~}\StringTok{ }\DecValTok{3}\NormalTok{, }
\NormalTok{             Phase2 }\OperatorTok{>=}\StringTok{ }\DecValTok{20} \OperatorTok{|}\StringTok{ }\NormalTok{rcsi23 }\OperatorTok{>=}\StringTok{ }\DecValTok{20} \OperatorTok{~}\StringTok{ }\DecValTok{2}\NormalTok{,}
             \OtherTok{TRUE} \OperatorTok{~}\StringTok{ }\DecValTok{1}\NormalTok{)) }\OperatorTok\StringTok{ }\KeywordTok{select}\NormalTok{(ADMIN1Name, ADMIN2Name, ADMIN2Code, }\DataTypeTok{rCSI_Phase1 =}\NormalTok{ Phase1, }\DataTypeTok{rCSI_Phase2 =}\NormalTok{ Phase2, }\DataTypeTok{rCSI_Phase3 =}\NormalTok{ Phase3, rCSI_finalphase)}
\end{Highlighting}
\end{Shaded}

Here is the R syntax file:

\href{https://github.com/WFP-VAM/RBD_FS_CH_guide_EN/blob/master/syntax/RBDstandardized_RsyntaxrCSI.R}{RBDstandardized\_RsyntaxrCSI}

\hypertarget{food-consumption-score-household-dietary-diversity}{%
\chapter{Food Consumption Score \& Household Dietary Diversity}\label{food-consumption-score-household-dietary-diversity}}

The modules below allow you to collect and calculate FCS, FCS-N and HDDS indicators.

\hypertarget{references-resources-2}{%
\section{References / Resources}\label{references-resources-2}}

\begin{enumerate}
\def\labelenumi{\arabic{enumi}.}
\item
  FCS - \url{https://documents.wfp.org/stellent/groups/public/documents/manual_guide_proced/wfp271449.pdf}
\item
  FCS\_N - \url{https://documents.wfp.org/stellent/groups/public/documents/manual_guide_proced/wfp277333.pdf}
\item
  HDDS - \url{http://www.fao.org/fileadmin/user_upload/wa_workshop/docs/FAO-guidelines-dietary-diversity2011.pdf}
\end{enumerate}

\hypertarget{standardized-questionnaire-2}{%
\section{Standardized Questionnaire}\label{standardized-questionnaire-2}}

\hypertarget{paper-version-of-questionnaire-2}{%
\subsection{Paper Version of Questionnaire}\label{paper-version-of-questionnaire-2}}

Here is the standardized module in word format:

\href{https://github.com/WFP-VAM/RBD_FS_CH_guide_EN/blob/master/questionnaires/RBDstandardized_questionnaireFCSHDDS.docx}{RBDstandardized\_questionnaireFCSHDDS}

\hypertarget{electronic-version-of-questionnaire-2}{%
\subsection{Electronic Version of Questionnaire}\label{electronic-version-of-questionnaire-2}}

Here is the standardized module in xlsform:

\href{https://github.com/WFP-VAM/RBD_FS_CH_guide_EN/blob/master/questionnaires/RBDstandardized_questionnaireFCSHDDS.xlsx}{RBDstandardized\_questionnaireFCSHDDS}

\hypertarget{calculation-of-fcs-hdds-indicators}{%
\section{Calculation of FCS \& HDDS Indicators}\label{calculation-of-fcs-hdds-indicators}}

\hypertarget{example-data-set-2}{%
\subsection{Example data set}\label{example-data-set-2}}

Here is the example data set:
\href{https://github.com/WFP-VAM/RBD_FS_CH_guide_EN/blob/master/example_datasets/dataFCSHDDSEng.sav}{dataFCSHDDSEng}

still need to create

\hypertarget{spss-syntax-2}{%
\subsection{SPSS Syntax}\label{spss-syntax-2}}

\begin{Shaded}
\begin{Highlighting}[]
\NormalTok{GET}
\NormalTok{  FILE=}\StringTok{'E:}\CharTok{\textbackslash{}e}\StringTok{xample_datasets\textbackslash{}dataFCSHDDSEng.sav'}\NormalTok{.}
\NormalTok{DATASET NAME DataSet1 WINDOW=FRONT.}

\OperatorTok{**}\StringTok{ }\NormalTok{calculate Food Consumption Score }

\NormalTok{compute FCS =}\StringTok{ }\KeywordTok{sum}\NormalTok{(FCSStap}\OperatorTok{*}\DecValTok{2}\NormalTok{, FCSPulse}\OperatorTok{*}\DecValTok{3}\NormalTok{, FCSDairy}\OperatorTok{*}\DecValTok{4}\NormalTok{, FCSPr}\OperatorTok{*}\DecValTok{4}\NormalTok{, FCSVeg, FCSFruit, FCSFat}\OperatorTok{*}\FloatTok{0.5}\NormalTok{, FCSSugar}\OperatorTok{*}\FloatTok{0.5}\NormalTok{).}
\NormalTok{variable labels FCS }\StringTok{"Food Consumption Score"}\NormalTok{.}

\OperatorTok{**}\StringTok{ }\NormalTok{create food consumption groups from food consumption score }\OperatorTok{-}\StringTok{ }\DecValTok{21}\OperatorTok{/}\DecValTok{35}\NormalTok{ and }\DecValTok{28}\OperatorTok{/}\DecValTok{42}\NormalTok{ thresholds}

\NormalTok{recode }\KeywordTok{FCS}\NormalTok{ (}\DecValTok{0}\NormalTok{ thru }\DecValTok{21}\NormalTok{ =}\StringTok{ }\DecValTok{1}\NormalTok{) (}\DecValTok{21}\NormalTok{ thru }\DecValTok{35}\NormalTok{ =}\StringTok{ }\DecValTok{2}\NormalTok{) (}\DecValTok{35}\NormalTok{ thru }\DataTypeTok{highest =} \DecValTok{3}\NormalTok{) into FCSCat21.}
\NormalTok{variable labels FCSCat21 }\StringTok{"Food Consumption Groups - 21/35 thresholds"}\NormalTok{.}
\NormalTok{recode }\KeywordTok{FCS}\NormalTok{ (}\DecValTok{0}\NormalTok{ thru }\DecValTok{28}\NormalTok{ =}\StringTok{ }\DecValTok{1}\NormalTok{) (}\DecValTok{28}\NormalTok{ thru }\DecValTok{42}\NormalTok{ =}\StringTok{ }\DecValTok{2}\NormalTok{) (}\DecValTok{42}\NormalTok{ thru }\DataTypeTok{highest =} \DecValTok{3}\NormalTok{) into FCSCat28.}
\NormalTok{variable labels FCSCat28  }\StringTok{"Food Consumption Groups - 28/42 thresholds"}\NormalTok{.}

\NormalTok{VALUE LABELS FCSCat21 FCSCat28 }
\DecValTok{1} \StringTok{"Poor"}
\DecValTok{2} \StringTok{"Borderline"}
\DecValTok{3} \StringTok{"Acceptable"}\NormalTok{.}

\OperatorTok{**}\StringTok{ }\NormalTok{calculate Household Dietary Diversity Score}

\NormalTok{compute HDDS =}\StringTok{ }\KeywordTok{sum}\NormalTok{(HDDSStapCer,HDDSStapRoot,HDDSPulse,HDDSDairy,HDDSPrMeatF,HDDSPrMeatO,HDDSPrFish,}
\NormalTok{HDDSPrEgg,HDDSVegOrg,HDDSVegGre,HDDSVegOth,HDDSFruitOrg,HDDSFruitOth,HDDSFat,HDDSSugar,HDDSCond).}
\NormalTok{variable labels HDDS }\StringTok{"Household Dietary Diversity Score"}\NormalTok{.}
\end{Highlighting}
\end{Shaded}

Here is the SPSS syntax file:

\href{https://github.com/WFP-VAM/RBD_FS_CH_guide_EN/blob/master/syntax/RBDstandardized_SPSSsyntaxFCSHDDS.sav}{RBDstandardized\_SPSSsyntaxFCSHDDS}

\hypertarget{r-syntax-2}{%
\subsection{R Syntax}\label{r-syntax-2}}

\begin{Shaded}
\begin{Highlighting}[]
\KeywordTok{library}\NormalTok{(haven)}
\KeywordTok{library}\NormalTok{(labelled)}
\KeywordTok{library}\NormalTok{(tidyverse)}

\CommentTok{#import dataset}
\NormalTok{dataFCSHDDSEng <-}\StringTok{ }\KeywordTok{read_sav}\NormalTok{(}\StringTok{"example_datasets/dataFCSHDDSEng.sav"}\NormalTok{)}
\CommentTok{#convert to labels}
\NormalTok{dataFCSHDDSEng <-}\StringTok{ }\KeywordTok{to_factor}\NormalTok{(dataFCSHDDSEng)}

\CommentTok{#calculate FCS}
\NormalTok{dataFCSHDDSEng <-}\StringTok{ }\NormalTok{dataFCSHDDSEng }\OperatorTok\StringTok{ }\KeywordTok{mutate}\NormalTok{(}\DataTypeTok{FCS =}\NormalTok{ (}\DecValTok{2} \OperatorTok{*}\StringTok{ }\NormalTok{FCSStap) }\OperatorTok{+}\StringTok{ }\NormalTok{(}\DecValTok{3} \OperatorTok{*}\StringTok{ }\NormalTok{FCSPulse)}\OperatorTok{+}\StringTok{ }\NormalTok{(}\DecValTok{4}\OperatorTok{*}\NormalTok{FCSPr) }\OperatorTok{+}\NormalTok{FCSVeg  }\OperatorTok{+}\NormalTok{FCSFruit }\OperatorTok{+}\NormalTok{(}\DecValTok{4}\OperatorTok{*}\NormalTok{FCSDairy) }\OperatorTok{+}\StringTok{ }\NormalTok{(}\FloatTok{0.5}\OperatorTok{*}\NormalTok{FCSFat) }\OperatorTok{+}\StringTok{ }\NormalTok{(}\FloatTok{0.5}\OperatorTok{*}\NormalTok{FCSSugar))}
\CommentTok{#create FCG groups based on 21/25 or 28/42 thresholds}
\NormalTok{dataFCSHDDSEng <-}\StringTok{ }\NormalTok{dataFCSHDDSEng }\OperatorTok\StringTok{ }\KeywordTok{mutate}\NormalTok{(}
  \DataTypeTok{FCSCat21 =} \KeywordTok{case_when}\NormalTok{(}
\NormalTok{  FCS }\OperatorTok{<=}\StringTok{ }\DecValTok{21} \OperatorTok{~}\StringTok{ "Poor"}\NormalTok{, }\KeywordTok{between}\NormalTok{(FCS, }\FloatTok{21.5}\NormalTok{, }\DecValTok{35}\NormalTok{) }\OperatorTok{~}\StringTok{ "Borderline"}\NormalTok{, FCS }\OperatorTok{>}\StringTok{ }\DecValTok{35} \OperatorTok{~}\StringTok{ "Acceptable"}\NormalTok{),}
  \DataTypeTok{FCSCat28 =} \KeywordTok{case_when}\NormalTok{(}
\NormalTok{  FCS }\OperatorTok{<=}\StringTok{ }\DecValTok{28} \OperatorTok{~}\StringTok{ "Poor"}\NormalTok{, }\KeywordTok{between}\NormalTok{(FCS, }\FloatTok{28.5}\NormalTok{, }\DecValTok{42}\NormalTok{) }\OperatorTok{~}\StringTok{ "Borderline"}\NormalTok{, FCS }\OperatorTok{>}\StringTok{ }\DecValTok{42} \OperatorTok{~}\StringTok{ "Acceptable"}\NormalTok{))}
\KeywordTok{var_label}\NormalTok{(dataFCSHDDSEng}\OperatorTok{$}\NormalTok{FCSCat21) <-}\StringTok{ "Food Consumption Group 21/35 thresholds"}
\KeywordTok{var_label}\NormalTok{(dataFCSHDDSEng}\OperatorTok{$}\NormalTok{FCSCat28) <-}\StringTok{  "Food Consumption Group 28/42 thresholds"}



\CommentTok{#calculate HDDS first by creating the 12 groups based on the 16 questions }
\NormalTok{dataFCSHDDSEng <-}\StringTok{ }\NormalTok{dataFCSHDDSEng }\OperatorTok\StringTok{ }\KeywordTok{mutate}\NormalTok{(}
    \DataTypeTok{HDDSStapCer =} \KeywordTok{case_when}\NormalTok{(HDDSStapCer }\OperatorTok{==}\StringTok{ "Yes"} \OperatorTok{~}\StringTok{ }\DecValTok{1}\NormalTok{, }\OtherTok{TRUE} \OperatorTok{~}\StringTok{ }\DecValTok{0}\NormalTok{),}
    \DataTypeTok{HDDSStapRoot =} \KeywordTok{case_when}\NormalTok{(HDDSStapRoot  }\OperatorTok{==}\StringTok{ "Yes"} \OperatorTok{~}\StringTok{ }\DecValTok{1}\NormalTok{, }\OtherTok{TRUE} \OperatorTok{~}\StringTok{ }\DecValTok{0}\NormalTok{),}
    \DataTypeTok{HDDSVeg =} \KeywordTok{case_when}\NormalTok{(HDDSVegOrg  }\OperatorTok{==}\StringTok{ "Yes"} \OperatorTok{|}\StringTok{ }\NormalTok{HDDSVegGre }\OperatorTok{==}\StringTok{ "Yes"} \OperatorTok{|}\StringTok{ }\NormalTok{HDDSVegOth }\OperatorTok{==}\StringTok{ "Yes"} \OperatorTok{~}\StringTok{ }\DecValTok{1}\NormalTok{, }\OtherTok{TRUE} \OperatorTok{~}\StringTok{ }\DecValTok{0}\NormalTok{),}
    \DataTypeTok{HDDSFruit =} \KeywordTok{case_when}\NormalTok{(HDDSFruitOrg }\OperatorTok{==}\StringTok{ "Yes"} \OperatorTok{|}\StringTok{ }\NormalTok{HDDSFruitOth }\OperatorTok{==}\StringTok{ "Yes"} \OperatorTok{~}\StringTok{ }\DecValTok{1}\NormalTok{, }\OtherTok{TRUE} \OperatorTok{~}\StringTok{ }\DecValTok{0}\NormalTok{),}
    \DataTypeTok{HDDSPrMeat =} \KeywordTok{case_when}\NormalTok{(HDDSPrMeatF }\OperatorTok{==}\StringTok{ "Yes"} \OperatorTok{|}\StringTok{ }\NormalTok{HDDSPrMeatO }\OperatorTok{==}\StringTok{ "Yes"} \OperatorTok{~}\StringTok{ }\DecValTok{1}\NormalTok{, }\OtherTok{TRUE} \OperatorTok{~}\StringTok{ }\DecValTok{0}\NormalTok{),}
    \DataTypeTok{HDDSPrEgg =} \KeywordTok{case_when}\NormalTok{(HDDSPrEgg  }\OperatorTok{==}\StringTok{ "Yes"} \OperatorTok{~}\StringTok{ }\DecValTok{1}\NormalTok{, }\OtherTok{TRUE} \OperatorTok{~}\StringTok{ }\DecValTok{0}\NormalTok{),}
    \DataTypeTok{HDDSPrFish =} \KeywordTok{case_when}\NormalTok{(HDDSPrFish }\OperatorTok{==}\StringTok{ "Yes"} \OperatorTok{~}\StringTok{ }\DecValTok{1}\NormalTok{, }\OtherTok{TRUE} \OperatorTok{~}\StringTok{ }\DecValTok{0}\NormalTok{),}
    \DataTypeTok{HDDSPulse =} \KeywordTok{case_when}\NormalTok{(HDDSPulse }\OperatorTok{==}\StringTok{ "Yes"} \OperatorTok{~}\StringTok{ }\DecValTok{1}\NormalTok{, }\OtherTok{TRUE} \OperatorTok{~}\StringTok{ }\DecValTok{0}\NormalTok{),}
    \DataTypeTok{HDDSDairy =} \KeywordTok{case_when}\NormalTok{(HDDSDairy }\OperatorTok{==}\StringTok{ "Yes"} \OperatorTok{~}\StringTok{ }\DecValTok{1}\NormalTok{, }\OtherTok{TRUE} \OperatorTok{~}\StringTok{ }\DecValTok{0}\NormalTok{),}
    \DataTypeTok{HDDSFat =} \KeywordTok{case_when}\NormalTok{(HDDSFat }\OperatorTok{==}\StringTok{ "Yes"} \OperatorTok{~}\StringTok{ }\DecValTok{1}\NormalTok{, }\OtherTok{TRUE} \OperatorTok{~}\StringTok{ }\DecValTok{0}\NormalTok{),}
    \DataTypeTok{HDDSSugar =} \KeywordTok{case_when}\NormalTok{(HDDSSugar }\OperatorTok{==}\StringTok{ "Yes"} \OperatorTok{~}\StringTok{ }\DecValTok{1}\NormalTok{, }\OtherTok{TRUE} \OperatorTok{~}\StringTok{ }\DecValTok{0}\NormalTok{),}
    \DataTypeTok{HDDSCond =} \KeywordTok{case_when}\NormalTok{(HDDSCond }\OperatorTok{==}\StringTok{ "Yes"}\OperatorTok{~}\StringTok{ }\DecValTok{1}\NormalTok{, }\OtherTok{TRUE} \OperatorTok{~}\StringTok{ }\DecValTok{0}\NormalTok{))}

\CommentTok{#Calculate HDDS and Cadre Harmonise Phases}
\NormalTok{dataFCSHDDSEng <-}\StringTok{ }\NormalTok{dataFCSHDDSEng }\OperatorTok\StringTok{ }\KeywordTok{mutate}\NormalTok{(}\DataTypeTok{HDDS =}\NormalTok{ HDDSStapCer }\OperatorTok{+}\NormalTok{HDDSStapRoot }\OperatorTok{+}\NormalTok{HDDSVeg }\OperatorTok{+}\NormalTok{HDDSFruit }\OperatorTok{+}\NormalTok{HDDSPrMeat }\OperatorTok{+}\NormalTok{HDDSPrEgg }\OperatorTok{+}\NormalTok{HDDSPrFish }\OperatorTok{+}\NormalTok{HDDSPulse }\OperatorTok{+}\NormalTok{HDDSDairy }\OperatorTok{+}\NormalTok{HDDSFat }\OperatorTok{+}\NormalTok{HDDSSugar }\OperatorTok{+}\NormalTok{HDDSCond)}
\KeywordTok{var_label}\NormalTok{(dataFCSHDDSEng}\OperatorTok{$}\NormalTok{HDDS) <-}\StringTok{ "Hosehold Dietary Diversity Score"}

\CommentTok{#Calucate Cadre Harmonise HDDS phasing categories }
\NormalTok{dataFCSHDDSEng <-}\StringTok{ }\NormalTok{dataFCSHDDSEng }\OperatorTok\StringTok{  }\KeywordTok{mutate}\NormalTok{(}\DataTypeTok{HDDS_CH =} \KeywordTok{case_when}\NormalTok{(}
\NormalTok{    HDDS }\OperatorTok{>=}\StringTok{ }\DecValTok{5} \OperatorTok{~}\StringTok{ "Phase1"}\NormalTok{, }
\NormalTok{    HDDS }\OperatorTok{==}\StringTok{ }\DecValTok{4} \OperatorTok{~}\StringTok{ "Phase2"}\NormalTok{,       }
\NormalTok{    HDDS }\OperatorTok{==}\StringTok{ }\DecValTok{3} \OperatorTok{~}\StringTok{ "Phase3"}\NormalTok{,}
\NormalTok{    HDDS }\OperatorTok{==}\StringTok{ }\DecValTok{2} \OperatorTok{~}\StringTok{ "Phase4"}\NormalTok{,}
\NormalTok{    HDDS }\OperatorTok{<}\StringTok{ }\DecValTok{2} \OperatorTok{~}\StringTok{ "Phase5"}\NormalTok{))}
\KeywordTok{var_label}\NormalTok{(dataFCSHDDSEng}\OperatorTok{$}\NormalTok{HDDS_CH) <-}\StringTok{ "Hosehold Dietary Diversity Score - CH phasing"}

\CommentTok{#Generate table of proportion of households in FCG by Adm1 and Adm2 using weights }
\CommentTok{#Food Consumption Group 21/35 cutoff }
\NormalTok{FCSCat_table_wide <-}\StringTok{ }\NormalTok{dataFCSHDDSEng }\OperatorTok\StringTok{ }
\StringTok{  }\KeywordTok{drop_na}\NormalTok{(FCSCat21) }\OperatorTok
\StringTok{  }\KeywordTok{group_by}\NormalTok{(ADMIN1Name, ADMIN2Name, ADMIN2Code) }\OperatorTok
\StringTok{  }\KeywordTok{count}\NormalTok{(FCSCat21, }\DataTypeTok{wt =}\NormalTok{ WeightHH) }\OperatorTok
\StringTok{  }\KeywordTok{mutate}\NormalTok{(}\DataTypeTok{perc =} \DecValTok{100} \OperatorTok{*}\StringTok{ }\NormalTok{n }\OperatorTok{/}\StringTok{ }\KeywordTok{sum}\NormalTok{(n)) }\OperatorTok
\StringTok{  }\KeywordTok{ungroup}\NormalTok{() }\OperatorTok\StringTok{ }\KeywordTok{select}\NormalTok{(}\OperatorTok{-}\NormalTok{n) }\OperatorTok
\StringTok{  }\KeywordTok{spread}\NormalTok{(}\DataTypeTok{key =}\NormalTok{ FCSCat21, }\DataTypeTok{value =}\NormalTok{ perc) }\OperatorTok\StringTok{ }\KeywordTok{replace}\NormalTok{(., }\KeywordTok{is.na}\NormalTok{(.), }\DecValTok{0}\NormalTok{) }\OperatorTok\StringTok{ }\KeywordTok{mutate_if}\NormalTok{(is.numeric, round, }\DecValTok{1}\NormalTok{)}

\CommentTok{#Calculate phasing of FCG indicator for area (applying CH 20% rules)}
\NormalTok{FCSCat_table_wide <-}\StringTok{ }\NormalTok{FCSCat_table_wide }\OperatorTok
\StringTok{  }\KeywordTok{mutate}\NormalTok{(}\DataTypeTok{PoorBorderline =}\NormalTok{ Poor }\OperatorTok{+}\StringTok{ }\NormalTok{Borderline, }\DataTypeTok{FCG_finalphase =} \KeywordTok{case_when}\NormalTok{(}
\NormalTok{  Poor }\OperatorTok{<}\StringTok{ }\DecValTok{5} \OperatorTok{~}\StringTok{ }\DecValTok{1}\NormalTok{,  }\CommentTok{#if less than 5% are in the poor food group then phase 1}
\NormalTok{  Poor }\OperatorTok{>=}\StringTok{ }\DecValTok{20} \OperatorTok{~}\StringTok{ }\DecValTok{4}\NormalTok{, }\CommentTok{#if 20% or more are in the poor food group then phase 4}
  \KeywordTok{between}\NormalTok{(Poor,}\DecValTok{5}\NormalTok{,}\DecValTok{10}\NormalTok{) }\OperatorTok{~}\StringTok{ }\DecValTok{2}\NormalTok{, }\CommentTok{#if % of people are between 5 and 10%  then phase2}
  \KeywordTok{between}\NormalTok{(Poor,}\DecValTok{10}\NormalTok{,}\DecValTok{20}\NormalTok{) }\OperatorTok{&}\StringTok{ }\NormalTok{PoorBorderline }\OperatorTok{<}\StringTok{ }\DecValTok{30} \OperatorTok{~}\StringTok{ }\DecValTok{2}\NormalTok{, }\CommentTok{#if % of people in poor food group are between 20 and 20% and the % of people who are in poor and borderline is less than 30 % then phase2}
  \KeywordTok{between}\NormalTok{(Poor,}\DecValTok{10}\NormalTok{,}\DecValTok{20}\NormalTok{) }\OperatorTok{&}\StringTok{ }\NormalTok{PoorBorderline }\OperatorTok{>=}\StringTok{ }\DecValTok{30} \OperatorTok{~}\StringTok{ }\DecValTok{3}\NormalTok{)) }\OperatorTok\StringTok{ }\CommentTok{#if % of people in poor food group are between 20 and 20% and the % of people who are in poor and borderline is less than 30 % then phase2}
\StringTok{  }\KeywordTok{select}\NormalTok{(ADMIN1Name, ADMIN2Name, ADMIN2Code, }\DataTypeTok{FCG_Poor =}\NormalTok{ Poor, }\DataTypeTok{FCG_Borderline =}\NormalTok{ Borderline, }\DataTypeTok{FCG_Acceptable =}\NormalTok{ Acceptable, FCG_finalphase) }\CommentTok{#select only relevant variables and order in proper sequence}

\CommentTok{#Generate table of proportion of households by HDDS by Adm1 and Adm2 using weights}
\NormalTok{CH_HDDS_table_wide <-}\StringTok{ }\NormalTok{dataFCSHDDSEng }\OperatorTok\StringTok{ }
\StringTok{  }\KeywordTok{drop_na}\NormalTok{(HDDS_CH) }\OperatorTok
\StringTok{  }\KeywordTok{group_by}\NormalTok{(ADMIN1Name, ADMIN2Name, ADMIN2Code) }\OperatorTok
\StringTok{  }\KeywordTok{count}\NormalTok{(HDDS_CH, }\DataTypeTok{wt =}\NormalTok{ WeightHH) }\OperatorTok
\StringTok{  }\KeywordTok{mutate}\NormalTok{(}\DataTypeTok{perc =} \DecValTok{100} \OperatorTok{*}\StringTok{ }\NormalTok{n }\OperatorTok{/}\StringTok{ }\KeywordTok{sum}\NormalTok{(n)) }\OperatorTok
\StringTok{  }\KeywordTok{ungroup}\NormalTok{() }\OperatorTok\StringTok{ }\KeywordTok{select}\NormalTok{(}\OperatorTok{-}\NormalTok{n) }\OperatorTok
\StringTok{  }\KeywordTok{spread}\NormalTok{(}\DataTypeTok{key =}\NormalTok{ HDDS_CH, }\DataTypeTok{value =}\NormalTok{ perc) }\OperatorTok\StringTok{ }\KeywordTok{replace}\NormalTok{(., }\KeywordTok{is.na}\NormalTok{(.), }\DecValTok{0}\NormalTok{) }\OperatorTok\StringTok{ }\KeywordTok{mutate_if}\NormalTok{(is.numeric, round, }\DecValTok{1}\NormalTok{)}

\CommentTok{#Calculate phasing of HDDS indicator for area (applying CH 20% rules)}


\CommentTok{#Apply the 20% rule (if it is 20% in that phase or the sum of higher phases equals 20%) }
\NormalTok{CH_HDDS_table_wide <-}\StringTok{ }\NormalTok{CH_HDDS_table_wide }\OperatorTok\StringTok{ }
\StringTok{    }\KeywordTok{mutate}\NormalTok{(}
    \DataTypeTok{phase2345 =} \StringTok{`}\DataTypeTok{Phase2}\StringTok{`} \OperatorTok{+}\StringTok{ `}\DataTypeTok{Phase3}\StringTok{`} \OperatorTok{+}\StringTok{ `}\DataTypeTok{Phase4}\StringTok{`} \OperatorTok{+}\StringTok{ `}\DataTypeTok{Phase5}\StringTok{`}\NormalTok{, }
    \DataTypeTok{phase345 =} \StringTok{`}\DataTypeTok{Phase3}\StringTok{`} \OperatorTok{+}\StringTok{ `}\DataTypeTok{Phase4}\StringTok{`} \OperatorTok{+}\StringTok{ `}\DataTypeTok{Phase5}\StringTok{`}\NormalTok{,}
    \DataTypeTok{phase45 =} \StringTok{`}\DataTypeTok{Phase4}\StringTok{`} \OperatorTok{+}\StringTok{ `}\DataTypeTok{Phase5}\StringTok{`}\NormalTok{, }
    \DataTypeTok{HDDS_finalphase =} \KeywordTok{case_when}\NormalTok{(}
    \StringTok{`}\DataTypeTok{Phase5}\StringTok{`} \OperatorTok{>=}\StringTok{ }\DecValTok{20} \OperatorTok{~}\StringTok{ }\DecValTok{5}\NormalTok{, }
    \StringTok{`}\DataTypeTok{Phase4}\StringTok{`} \OperatorTok{>=}\StringTok{ }\DecValTok{20} \OperatorTok{|}\StringTok{ }\NormalTok{phase45 }\OperatorTok{>=}\StringTok{ }\DecValTok{20} \OperatorTok{~}\StringTok{ }\DecValTok{4}\NormalTok{, }
    \StringTok{`}\DataTypeTok{Phase3}\StringTok{`} \OperatorTok{>=}\StringTok{ }\DecValTok{20} \OperatorTok{|}\StringTok{ }\NormalTok{phase345 }\OperatorTok{>=}\StringTok{ }\DecValTok{20} \OperatorTok{~}\StringTok{ }\DecValTok{3}\NormalTok{, }
    \StringTok{`}\DataTypeTok{Phase2}\StringTok{`} \OperatorTok{>=}\StringTok{ }\DecValTok{20} \OperatorTok{|}\StringTok{ }\NormalTok{phase2345 }\OperatorTok{>=}\StringTok{ }\DecValTok{20} \OperatorTok{~}\StringTok{ }\DecValTok{2}\NormalTok{,}
     \OtherTok{TRUE} \OperatorTok{~}\StringTok{ }\DecValTok{1}\NormalTok{)) }\OperatorTok\StringTok{ }
\StringTok{  }\KeywordTok{select}\NormalTok{(ADMIN1Name, ADMIN2Name, ADMIN2Code, }\DataTypeTok{HDDS_Phase1 =}\NormalTok{ Phase1, }\DataTypeTok{HDDS_Phase2 =}\NormalTok{ Phase2, }\DataTypeTok{HDDS_Phase3 =}\NormalTok{ Phase3, }\DataTypeTok{HDDS_Phase4 =}\NormalTok{ Phase4, }\DataTypeTok{HDDS_Phase5 =}\NormalTok{ Phase5, HDDS_finalphase) }
\end{Highlighting}
\end{Shaded}

Here is the R syntax file:

\href{https://github.com/WFP-VAM/RBD_FS_CH_guide_EN/blob/master/syntax/RBDstandardized_RsyntaxFCSHDDS.R}{RBDstandardized\_RsyntaxFCSHDDS}

\hypertarget{livelihood-coping-strategies}{%
\chapter{Livelihood Coping Strategies}\label{livelihood-coping-strategies}}

\hypertarget{references-resources-3}{%
\section{References / Resources}\label{references-resources-3}}

\begin{enumerate}
\def\labelenumi{\arabic{enumi}.}
\item
  Standard Livelihood Coping Strategies - \url{https://documents.wfp.org/stellent/groups/public/documents/manual_guide_proced/wfp271449.pdf}
\item
  ENA Livelihood Coping Strategies - \url{https://docs.wfp.org/api/documents/WFP-0000074197/download/}?
\end{enumerate}

\hypertarget{standardized-light-cari-module}{%
\section{Standardized Light CARI Module}\label{standardized-light-cari-module}}

\begin{longtable}[]{@{}lll@{}}
\toprule
\begin{minipage}[b]{0.15\columnwidth}\raggedright
Variable Name\strut
\end{minipage} & \begin{minipage}[b]{0.49\columnwidth}\raggedright
Question Label\strut
\end{minipage} & \begin{minipage}[b]{0.27\columnwidth}\raggedright
Answer Choices\strut
\end{minipage}\tabularnewline
\midrule
\endhead
\begin{minipage}[t]{0.15\columnwidth}\raggedright
LhCSIStress1\strut
\end{minipage} & \begin{minipage}[t]{0.49\columnwidth}\raggedright
During the past 30 days, did anyone in your household have to: \emph{Sell household assets/goods radio, furniture, refrigerator, television, jewelry etc..)} due to a lack of food or a lack of money to buy food?\strut
\end{minipage} & \begin{minipage}[t]{0.27\columnwidth}\raggedright
1) No, because I did not face a shortage of food 2) No; because I already sold those assets or did this activity in the last 12 months and cannot continue to do it 3) Yes 4) Not applicable\strut
\end{minipage}\tabularnewline
\begin{minipage}[t]{0.15\columnwidth}\raggedright
LhCSIStress2\strut
\end{minipage} & \begin{minipage}[t]{0.49\columnwidth}\raggedright
During the past 30 days, did anyone in your household have to: \emph{sell more animals (non-productive) than usual} due to a lack of food or a lack of money to buy food?\strut
\end{minipage} & \begin{minipage}[t]{0.27\columnwidth}\raggedright
1) No, because I did not face a shortage of food 2) No; because I already sold those assets or did this activity in the last 12 months and cannot continue to do it 3) Yes 4) Not applicable\strut
\end{minipage}\tabularnewline
\begin{minipage}[t]{0.15\columnwidth}\raggedright
LhCSIStress3\strut
\end{minipage} & \begin{minipage}[t]{0.49\columnwidth}\raggedright
During the past 30 days, did anyone in your household have to: \emph{spend savings} due to a lack of food or a lack of money to buy food?\strut
\end{minipage} & \begin{minipage}[t]{0.27\columnwidth}\raggedright
1) No, because I did not face a shortage of food 2) No; because I already sold those assets or did this activity in the last 12 months and cannot continue to do it 3) Yes 4) Not applicable\strut
\end{minipage}\tabularnewline
\begin{minipage}[t]{0.15\columnwidth}\raggedright
LhCSIStress4\strut
\end{minipage} & \begin{minipage}[t]{0.49\columnwidth}\raggedright
During the past 30 days, did anyone in your household have to: \emph{Borrow money / food from a formal lender / bank} due to a lack of food or a lack of money to buy food?\strut
\end{minipage} & \begin{minipage}[t]{0.27\columnwidth}\raggedright
1) No, because I did not face a shortage of food 2) No; because I already sold those assets or did this activity in the last 12 months and cannot continue to do it 3) Yes 4) Not applicable\strut
\end{minipage}\tabularnewline
\begin{minipage}[t]{0.15\columnwidth}\raggedright
LhCSICrisis1\strut
\end{minipage} & \begin{minipage}[t]{0.49\columnwidth}\raggedright
During the past 30 days, did anyone in your household have to: \emph{Reduce non-food expenses on health (including drugs) and education } due to a lack of food or a lack of money to buy food?\strut
\end{minipage} & \begin{minipage}[t]{0.27\columnwidth}\raggedright
1) No, because I did not face a shortage of food 2) No; because I already sold those assets or did this activity in the last 12 months and cannot continue to do it 3) Yes 4) Not applicable\strut
\end{minipage}\tabularnewline
\begin{minipage}[t]{0.15\columnwidth}\raggedright
LhCSICrisis2\strut
\end{minipage} & \begin{minipage}[t]{0.49\columnwidth}\raggedright
During the past 30 days, did anyone in your household have to: \emph{sell productive assets or means of transport (sewing machine, wheelbarrow, bicycle, car, etc..)} due to a lack of food or a lack of money to buy food?\strut
\end{minipage} & \begin{minipage}[t]{0.27\columnwidth}\raggedright
1) No, because I did not face a shortage of food 2) No; because I already sold those assets or did this activity in the last 12 months and cannot continue to do it 3) Yes 4) Not applicable\strut
\end{minipage}\tabularnewline
\begin{minipage}[t]{0.15\columnwidth}\raggedright
LhCSICrisis3\strut
\end{minipage} & \begin{minipage}[t]{0.49\columnwidth}\raggedright
During the past 30 days, did anyone in your household have to: \emph{withdraw children from school} due to a lack of food or a lack of money to buy food?\strut
\end{minipage} & \begin{minipage}[t]{0.27\columnwidth}\raggedright
1) No, because I did not face a shortage of food 2) No; because I already sold those assets or did this activity in the last 12 months and cannot continue to do it 3) Yes 4) Not applicable\strut
\end{minipage}\tabularnewline
\begin{minipage}[t]{0.15\columnwidth}\raggedright
LhCSIEmergency1\strut
\end{minipage} & \begin{minipage}[t]{0.49\columnwidth}\raggedright
During the past 30 days, did anyone in your household have to: \emph{sell house or land} due to a lack of food or a lack of money to buy food?\strut
\end{minipage} & \begin{minipage}[t]{0.27\columnwidth}\raggedright
1) No, because I did not face a shortage of food 2) No; because I already sold those assets or did this activity in the last 12 months and cannot continue to do it 3) Yes 4) Not applicable\strut
\end{minipage}\tabularnewline
\begin{minipage}[t]{0.15\columnwidth}\raggedright
LhCSIEmergency2\strut
\end{minipage} & \begin{minipage}[t]{0.49\columnwidth}\raggedright
During the past 30 days, did anyone in your household have to: \emph{sell last female animals} due to a lack of food or a lack of money to buy food?\strut
\end{minipage} & \begin{minipage}[t]{0.27\columnwidth}\raggedright
1) No, because I did not face a shortage of food 2) No; because I already sold those assets or did this activity in the last 12 months and cannot continue to do it 3) Yes 4) Not applicable\strut
\end{minipage}\tabularnewline
\begin{minipage}[t]{0.15\columnwidth}\raggedright
LhCSIEmergency3\strut
\end{minipage} & \begin{minipage}[t]{0.49\columnwidth}\raggedright
During the past 30 days, did anyone in your household have to: \emph{beg } due to a lack of food or a lack of money to buy food?\strut
\end{minipage} & \begin{minipage}[t]{0.27\columnwidth}\raggedright
1) No, because I did not face a shortage of food 2) No; because I already sold those assets or did this activity in the last 12 months and cannot continue to do it 3) Yes 4) Not applicable\strut
\end{minipage}\tabularnewline
\bottomrule
\end{longtable}

\hypertarget{paper-version-of-questionnaire-3}{%
\subsection{Paper Version of Questionnaire}\label{paper-version-of-questionnaire-3}}

Here is the standardized module in word format:
\href{https://github.com/WFP-VAM/RBD_FS_CH_guide_EN/blob/master/questionnaires/RBDstandardized_questionnaireLHCS_CARI.docx}{RBDstandardized\_questionnaireLHCS\_CARI}

\hypertarget{electronic-version-of-questionnaire-3}{%
\subsection{Electronic Version of Questionnaire}\label{electronic-version-of-questionnaire-3}}

Here is the standardized module in xlsform:

\href{https://github.com/WFP-VAM/RBD_FS_CH_guide_EN/blob/master/questionnaires/RBDstandardized_questionnaireLHCS_CARI.xlsx}{RBDstandardized\_questionnaireLHCS\_CARI}

\hypertarget{analysis}{%
\subsection{Analysis}\label{analysis}}

\hypertarget{example-data-set-3}{%
\subsubsection{Example data set}\label{example-data-set-3}}

\href{https://github.com/WFP-VAM/RBD_FS_CH_guide_EN/blob/master/example_datasets/dataLHCS_CARIEng.sav}{dataLHCS\_CARIEng}

Here is the example data set:

\hypertarget{spss-syntax-3}{%
\subsubsection{SPSS Syntax}\label{spss-syntax-3}}

\begin{Shaded}
\begin{Highlighting}[]
\NormalTok{not availible yet}
\end{Highlighting}
\end{Shaded}

Here is the SPSS syntax file:

still need to do

\hypertarget{r-syntax-3}{%
\subsubsection{R Syntax}\label{r-syntax-3}}

\begin{Shaded}
\begin{Highlighting}[]
\KeywordTok{library}\NormalTok{(haven)}
\KeywordTok{library}\NormalTok{(labelled)}
\KeywordTok{library}\NormalTok{(tidyverse)}

\CommentTok{#import dataset}
\NormalTok{dataLHCS_CARIEng <-}\StringTok{ }\KeywordTok{read_sav}\NormalTok{(}\StringTok{"example_datasets/dataLHCS_CARIEng.sav"}\NormalTok{)}

\CommentTok{#Calculate HHS }
\NormalTok{dataLHCS_CARIEng<-}\StringTok{ }\KeywordTok{to_factor}\NormalTok{(dataLHCS_CARIEng)}

\CommentTok{#create a variable to specify if the household used any of the strategies by severity }
\CommentTok{#stress}
\NormalTok{dataLHCS_CARIEng <-}\StringTok{ }\NormalTok{dataLHCS_CARIEng }\OperatorTok\StringTok{ }\KeywordTok{mutate}\NormalTok{(}\DataTypeTok{stress_coping =} \KeywordTok{case_when}\NormalTok{(}
\NormalTok{LhCSIStress1 }\OperatorTok\StringTok{ }\KeywordTok{c}\NormalTok{(}\StringTok{"Yes"}\NormalTok{,}\StringTok{"No; because I already sold those assets or did this activity in the last 12 months and cannot continue to do it"}\NormalTok{) }\OperatorTok{~}\StringTok{ "usedstress"}\NormalTok{,}
\NormalTok{LhCSIStress2 }\OperatorTok\StringTok{ }\KeywordTok{c}\NormalTok{(}\StringTok{"Yes"}\NormalTok{,}\StringTok{"No; because I already sold those assets or did this activity in the last 12 months and cannot continue to do it"}\NormalTok{) }\OperatorTok{~}\StringTok{ "usedstress"}\NormalTok{,}
\NormalTok{LhCSIStress3 }\OperatorTok\StringTok{ }\KeywordTok{c}\NormalTok{(}\StringTok{"Yes"}\NormalTok{,}\StringTok{"No; because I already sold those assets or did this activity in the last 12 months and cannot continue to do it"}\NormalTok{) }\OperatorTok{~}\StringTok{ "usedstress"}\NormalTok{,}
\NormalTok{LhCSIStress4 }\OperatorTok\StringTok{ }\KeywordTok{c}\NormalTok{(}\StringTok{"Yes"}\NormalTok{,}\StringTok{"No; because I already sold those assets or did this activity in the last 12 months and cannot continue to do it"}\NormalTok{) }\OperatorTok{~}\StringTok{ "usedstress"}\NormalTok{,}
\OtherTok{TRUE} \OperatorTok{~}\StringTok{ "notusedstress"}\NormalTok{))}
\CommentTok{#Crisis}
\NormalTok{dataLHCS_CARIEng <-}\StringTok{ }\NormalTok{dataLHCS_CARIEng }\OperatorTok\StringTok{ }\KeywordTok{mutate}\NormalTok{(}\DataTypeTok{crisis_coping =} \KeywordTok{case_when}\NormalTok{(}
\NormalTok{LhCSICrisis1 }\OperatorTok\StringTok{ }\KeywordTok{c}\NormalTok{(}\StringTok{"Yes"}\NormalTok{,}\StringTok{"No; because I already sold those assets or did this activity in the last 12 months and cannot continue to do it"}\NormalTok{) }\OperatorTok{~}\StringTok{ "usedcrisis"}\NormalTok{,}
\NormalTok{LhCSICrisis2 }\OperatorTok\StringTok{ }\KeywordTok{c}\NormalTok{(}\StringTok{"Yes"}\NormalTok{,}\StringTok{"No; because I already sold those assets or did this activity in the last 12 months and cannot continue to do it"}\NormalTok{) }\OperatorTok{~}\StringTok{ "usedcrisis"}\NormalTok{,}
\NormalTok{LhCSICrisis3 }\OperatorTok\StringTok{ }\KeywordTok{c}\NormalTok{(}\StringTok{"Yes"}\NormalTok{,}\StringTok{"No; because I already sold those assets or did this activity in the last 12 months and cannot continue to do it"}\NormalTok{) }\OperatorTok{~}\StringTok{ "usedcrisis"}\NormalTok{,}
\OtherTok{TRUE} \OperatorTok{~}\StringTok{ "notusedcrisis"}\NormalTok{))}
\CommentTok{#Emergency}
\NormalTok{dataLHCS_CARIEng <-}\StringTok{ }\NormalTok{dataLHCS_CARIEng }\OperatorTok\StringTok{ }\KeywordTok{mutate}\NormalTok{(}\DataTypeTok{emergency_coping =} \KeywordTok{case_when}\NormalTok{(}
\NormalTok{LhCSIEmergency1 }\OperatorTok\StringTok{ }\KeywordTok{c}\NormalTok{(}\StringTok{"Yes"}\NormalTok{,}\StringTok{"No; because I already sold those assets or did this activity in the last 12 months and cannot continue to do it"}\NormalTok{) }\OperatorTok{~}\StringTok{ "usedemergency"}\NormalTok{,}
\NormalTok{LhCSIEmergency2 }\OperatorTok\StringTok{ }\KeywordTok{c}\NormalTok{(}\StringTok{"Yes"}\NormalTok{,}\StringTok{"No; because I already sold those assets or did this activity in the last 12 months and cannot continue to do it"}\NormalTok{) }\OperatorTok{~}\StringTok{ "usedemergency"}\NormalTok{,}
\NormalTok{LhCSIEmergency3 }\OperatorTok\StringTok{ }\KeywordTok{c}\NormalTok{(}\StringTok{"Yes"}\NormalTok{,}\StringTok{"No; because I already sold those assets or did this activity in the last 12 months and cannot continue to do it"}\NormalTok{) }\OperatorTok{~}\StringTok{ "usedemergency"}\NormalTok{,}
\OtherTok{TRUE} \OperatorTok{~}\StringTok{ "noutusedemergency"}\NormalTok{))}

\CommentTok{#calculate Max_coping_behaviour}
\NormalTok{dataLHCS_CARIEng <-}\StringTok{ }\NormalTok{dataLHCS_CARIEng }\OperatorTok\StringTok{ }\KeywordTok{mutate}\NormalTok{(}\DataTypeTok{LhCSICat =} \KeywordTok{case_when}\NormalTok{(}
\NormalTok{  emergency_coping }\OperatorTok{==}\StringTok{ "usedemergency"} \OperatorTok{~}\StringTok{ "EmergencyStrategies"}\NormalTok{,}
\NormalTok{  crisis_coping }\OperatorTok{==}\StringTok{ "usedcrisis"} \OperatorTok{~}\StringTok{ "CrisisStrategies"}\NormalTok{,}
\NormalTok{  stress_coping }\OperatorTok{==}\StringTok{ "usedstress"} \OperatorTok{~}\StringTok{ "StressStrategies"}\NormalTok{,}
  \OtherTok{TRUE} \OperatorTok{~}\StringTok{ "NoStrategies"}\NormalTok{))}
\KeywordTok{var_label}\NormalTok{(dataLHCS_CARIEng}\OperatorTok{$}\NormalTok{LhCSICat) <-}\StringTok{ "Livelihood Coping Strategy categories - CARI light version"}



\CommentTok{#Generate table of proportion of households in CH LhCHS phases by Adm1 and Adm2 using weights}
\CommentTok{#Livelihood Coping Strategies }
\NormalTok{LhHCSCat_table_wide <-}\StringTok{ }\NormalTok{dataLHCS_CARIEng }\OperatorTok\StringTok{ }
\StringTok{  }\KeywordTok{drop_na}\NormalTok{(LhCSICat) }\OperatorTok
\StringTok{  }\KeywordTok{group_by}\NormalTok{(ADMIN1Name, ADMIN2Name) }\OperatorTok
\StringTok{  }\KeywordTok{count}\NormalTok{(LhCSICat, }\DataTypeTok{wt =}\NormalTok{ WeightHH) }\OperatorTok
\StringTok{  }\KeywordTok{mutate}\NormalTok{(}\DataTypeTok{perc =} \DecValTok{100} \OperatorTok{*}\StringTok{ }\NormalTok{n }\OperatorTok{/}\StringTok{ }\KeywordTok{sum}\NormalTok{(n)) }\OperatorTok
\StringTok{  }\KeywordTok{ungroup}\NormalTok{() }\OperatorTok\StringTok{ }\KeywordTok{select}\NormalTok{(}\OperatorTok{-}\NormalTok{n) }\OperatorTok
\StringTok{  }\KeywordTok{spread}\NormalTok{(}\DataTypeTok{key =}\NormalTok{ LhCSICat, }\DataTypeTok{value =}\NormalTok{ perc) }\OperatorTok\StringTok{ }\KeywordTok{replace}\NormalTok{(., }\KeywordTok{is.na}\NormalTok{(.), }\DecValTok{0}\NormalTok{) }\OperatorTok\StringTok{ }\KeywordTok{mutate_if}\NormalTok{(is.numeric, round, }\DecValTok{1}\NormalTok{)}
\end{Highlighting}
\end{Shaded}

Here is the R syntax file:

\href{https://github.com/WFP-VAM/RBD_FS_CH_guide_EN/blob/master/syntax/RBDstandardized_RsyntaxLHCS_CARI.R}{RBDstandardized\_RsyntaxLHCS\_CARI}

\hypertarget{standardized-module---essential-needs-assessment-ena}{%
\section{Standardized Module - Essential Needs Assessment (ENA)}\label{standardized-module---essential-needs-assessment-ena}}

\begin{longtable}[]{@{}lll@{}}
\toprule
\begin{minipage}[b]{0.11\columnwidth}\raggedright
Variable Name\strut
\end{minipage} & \begin{minipage}[b]{0.37\columnwidth}\raggedright
Question Label\strut
\end{minipage} & \begin{minipage}[b]{0.43\columnwidth}\raggedright
Answer Choices\strut
\end{minipage}\tabularnewline
\midrule
\endhead
\begin{minipage}[t]{0.11\columnwidth}\raggedright
LhCSIStress1\_EN\strut
\end{minipage} & \begin{minipage}[t]{0.37\columnwidth}\raggedright
During the past 30 days, did anyone in your household have to: \emph{Sell household assets/goods (radio, furniture, refrigerator, television, jewelry etc..)} because there was not enough resources (food, cash, else) to meet essential needs (e.g.~adequate shelter, education services, health services, food, etc)?\strut
\end{minipage} & \begin{minipage}[t]{0.43\columnwidth}\raggedright
1) No, because I did not face a shortage of food 2) No; because I already sold those assets or did this activity in the last 12 months and cannot continue to do it 3) Yes 4) Not applicable\strut
\end{minipage}\tabularnewline
\begin{minipage}[t]{0.11\columnwidth}\raggedright
LhCSIStress1\_EN\_why\strut
\end{minipage} & \begin{minipage}[t]{0.37\columnwidth}\raggedright
What was the MAIN REASON you or other members in your household adopted these coping strategies to access essential needs?\strut
\end{minipage} & \begin{minipage}[t]{0.43\columnwidth}\raggedright
1)To access food 2)To access education services/commodities (e.g.~uniforms, books) 3)To access health services/medicines 4)To buy and repair cloths or shoes 5)To access adequate shelter 6)To access water/sanitation facilities 7)To access essential dwelling services (electricity, energy, waste disposal\ldots) 8)To pay for existing debts 9)Other\strut
\end{minipage}\tabularnewline
\begin{minipage}[t]{0.11\columnwidth}\raggedright
LhCSIStress2\_EN\strut
\end{minipage} & \begin{minipage}[t]{0.37\columnwidth}\raggedright
During the past 30 days, did anyone in your household have to: \emph{Purchase food on credit or borrow food} because there was not enough resources (food, cash, else) to meet essential needs (e.g.~adequate shelter, education services, health services, food, etc)?\strut
\end{minipage} & \begin{minipage}[t]{0.43\columnwidth}\raggedright
1) No, because I did not face a shortage of food 2) No; because I already sold those assets or did this activity in the last 12 months and cannot continue to do it 3) Yes 4) Not applicable\strut
\end{minipage}\tabularnewline
\begin{minipage}[t]{0.11\columnwidth}\raggedright
LhCSIStress2\_EN\_why\strut
\end{minipage} & \begin{minipage}[t]{0.37\columnwidth}\raggedright
What was the MAIN REASON you or other members in your household adopted these coping strategies to access essential needs?\strut
\end{minipage} & \begin{minipage}[t]{0.43\columnwidth}\raggedright
1)To access food 2)To access education services/commodities (e.g.~uniforms, books) 3)To access health services/medicines 4)To buy and repair cloths or shoes 5)To access adequate shelter 6)To access water/sanitation facilities 7)To access essential dwelling services (electricity, energy, waste disposal\ldots) 8)To pay for existing debts 9)Other\strut
\end{minipage}\tabularnewline
\begin{minipage}[t]{0.11\columnwidth}\raggedright
LhCSIStress3\_EN\strut
\end{minipage} & \begin{minipage}[t]{0.37\columnwidth}\raggedright
During the past 30 days, did anyone in your household have to: \emph{spend savings} because there was not enough resources (food, cash, else) to meet essential needs (e.g.~adequate shelter, education services, health services, food, etc)?\strut
\end{minipage} & \begin{minipage}[t]{0.43\columnwidth}\raggedright
1) No, because I did not face a shortage of food 2) No; because I already sold those assets or did this activity in the last 12 months and cannot continue to do it 3) Yes 4) Not applicable\strut
\end{minipage}\tabularnewline
\begin{minipage}[t]{0.11\columnwidth}\raggedright
LhCSIStress3\_EN\_why\strut
\end{minipage} & \begin{minipage}[t]{0.37\columnwidth}\raggedright
What was the MAIN REASON you or other members in your household adopted these coping strategies to access essential needs?\strut
\end{minipage} & \begin{minipage}[t]{0.43\columnwidth}\raggedright
1)To access food 2)To access education services/commodities (e.g.~uniforms, books) 3)To access health services/medicines 4)To buy and repair cloths or shoes 5)To access adequate shelter 6)To access water/sanitation facilities 7)To access essential dwelling services (electricity, energy, waste disposal\ldots) 8)To pay for existing debts 9)Other\strut
\end{minipage}\tabularnewline
\begin{minipage}[t]{0.11\columnwidth}\raggedright
LhCSIStress4\_EN\strut
\end{minipage} & \begin{minipage}[t]{0.37\columnwidth}\raggedright
During the past 30 days, did anyone in your household have to: \emph{Borrow money} because there was not enough resources (food, cash, else) to meet essential needs (e.g.~adequate shelter, education services, health services, food, etc)?\strut
\end{minipage} & \begin{minipage}[t]{0.43\columnwidth}\raggedright
1) No, because I did not face a shortage of food 2) No; because I already sold those assets or did this activity in the last 12 months and cannot continue to do it 3) Yes 4) Not applicable\strut
\end{minipage}\tabularnewline
\begin{minipage}[t]{0.11\columnwidth}\raggedright
LhCSIStress4\_EN\_why\strut
\end{minipage} & \begin{minipage}[t]{0.37\columnwidth}\raggedright
What was the MAIN REASON you or other members in your household adopted these coping strategies to access essential needs?\strut
\end{minipage} & \begin{minipage}[t]{0.43\columnwidth}\raggedright
1)To access food 2)To access education services/commodities (e.g.~uniforms, books) 3)To access health services/medicines 4)To buy and repair cloths or shoes 5)To access adequate shelter 6)To access water/sanitation facilities 7)To access essential dwelling services (electricity, energy, waste disposal\ldots) 8)To pay for existing debts 9)Other\strut
\end{minipage}\tabularnewline
\begin{minipage}[t]{0.11\columnwidth}\raggedright
LhCSICrisis1\_EN\strut
\end{minipage} & \begin{minipage}[t]{0.37\columnwidth}\raggedright
During the past 30 days, did anyone in your household have to: \emph{sell productive assets or means of transport (sewing machine, wheelbarrow, bicycle, car, etc..)} because there was not enough resources (food, cash, else) to meet essential needs (e.g.~adequate shelter, education services, health services, food, etc)?\strut
\end{minipage} & \begin{minipage}[t]{0.43\columnwidth}\raggedright
1) No, because I did not face a shortage of food 2) No; because I already sold those assets or did this activity in the last 12 months and cannot continue to do it 3) Yes 4) Not applicable\strut
\end{minipage}\tabularnewline
\begin{minipage}[t]{0.11\columnwidth}\raggedright
LhCSICrisis1\_EN\_why\strut
\end{minipage} & \begin{minipage}[t]{0.37\columnwidth}\raggedright
What was the MAIN REASON you or other members in your household adopted these coping strategies to access essential needs?\strut
\end{minipage} & \begin{minipage}[t]{0.43\columnwidth}\raggedright
1)To access food 2)To access education services/commodities (e.g.~uniforms, books) 3)To access health services/medicines 4)To buy and repair cloths or shoes 5)To access adequate shelter 6)To access water/sanitation facilities 7)To access essential dwelling services (electricity, energy, waste disposal\ldots) 8)To pay for existing debts 9)Other\strut
\end{minipage}\tabularnewline
\begin{minipage}[t]{0.11\columnwidth}\raggedright
LhCSICrisis2\_EN\strut
\end{minipage} & \begin{minipage}[t]{0.37\columnwidth}\raggedright
During the past 30 days, did anyone in your household have to: \emph{consume seed stocks that were to be held/saved for the next season} because there was not enough resources (food, cash, else) to meet essential needs (e.g.~adequate shelter, education services, health services, food, etc)?\strut
\end{minipage} & \begin{minipage}[t]{0.43\columnwidth}\raggedright
1) No, because I did not face a shortage of food 2) No; because I already sold those assets or did this activity in the last 12 months and cannot continue to do it 3) Yes 4) Not applicable\strut
\end{minipage}\tabularnewline
\begin{minipage}[t]{0.11\columnwidth}\raggedright
LhCSICrisis2\_EN\_why\strut
\end{minipage} & \begin{minipage}[t]{0.37\columnwidth}\raggedright
What was the MAIN REASON you or other members in your household adopted these coping strategies to access essential needs?\strut
\end{minipage} & \begin{minipage}[t]{0.43\columnwidth}\raggedright
1)To access food 2)To access education services/commodities (e.g.~uniforms, books) 3)To access health services/medicines 4)To buy and repair cloths or shoes 5)To access adequate shelter 6)To access water/sanitation facilities 7)To access essential dwelling services (electricity, energy, waste disposal\ldots) 8)To pay for existing debts 9)Other\strut
\end{minipage}\tabularnewline
\begin{minipage}[t]{0.11\columnwidth}\raggedright
LhCSICrisis3\_EN\strut
\end{minipage} & \begin{minipage}[t]{0.37\columnwidth}\raggedright
During the past 30 days, did anyone in your household have to: \emph{Withdraw children from school} because there was not enough resources (food, cash, else) to meet essential needs (e.g.~adequate shelter, education services, health services, food, etc)?\strut
\end{minipage} & \begin{minipage}[t]{0.43\columnwidth}\raggedright
1) No, because I did not face a shortage of food 2) No; because I already sold those assets or did this activity in the last 12 months and cannot continue to do it 3) Yes 4) Not applicable\strut
\end{minipage}\tabularnewline
\begin{minipage}[t]{0.11\columnwidth}\raggedright
LhCSICrisis3\_EN\_why\strut
\end{minipage} & \begin{minipage}[t]{0.37\columnwidth}\raggedright
What was the MAIN REASON you or other members in your household adopted these coping strategies to access essential needs?\strut
\end{minipage} & \begin{minipage}[t]{0.43\columnwidth}\raggedright
1)To access food 2)To access education services/commodities (e.g.~uniforms, books) 3)To access health services/medicines 4)To buy and repair cloths or shoes 5)To access adequate shelter 6)To access water/sanitation facilities 7)To access essential dwelling services (electricity, energy, waste disposal\ldots) 8)To pay for existing debts 9)Other\strut
\end{minipage}\tabularnewline
\begin{minipage}[t]{0.11\columnwidth}\raggedright
LhCSIEmergency1\_EN\strut
\end{minipage} & \begin{minipage}[t]{0.37\columnwidth}\raggedright
During the past 30 days, did anyone in your household have to: \emph{Sell house or land because there was not enough resources (food, cash, else)} because there was not enough resources (food, cash, else) to meet essential needs (e.g.~adequate shelter, education services, health services, food, etc)?\strut
\end{minipage} & \begin{minipage}[t]{0.43\columnwidth}\raggedright
1) No, because I did not face a shortage of food 2) No; because I already sold those assets or did this activity in the last 12 months and cannot continue to do it 3) Yes 4) Not applicable\strut
\end{minipage}\tabularnewline
\begin{minipage}[t]{0.11\columnwidth}\raggedright
LhCSIEmergency1\_EN\_why\strut
\end{minipage} & \begin{minipage}[t]{0.37\columnwidth}\raggedright
What was the MAIN REASON you or other members in your household adopted these coping strategies to access essential needs?\strut
\end{minipage} & \begin{minipage}[t]{0.43\columnwidth}\raggedright
1)To access food 2)To access education services/commodities (e.g.~uniforms, books) 3)To access health services/medicines 4)To buy and repair cloths or shoes 5)To access adequate shelter 6)To access water/sanitation facilities 7)To access essential dwelling services (electricity, energy, waste disposal\ldots) 8)To pay for existing debts 9)Other\strut
\end{minipage}\tabularnewline
\begin{minipage}[t]{0.11\columnwidth}\raggedright
LhCSIEmergency2\_EN\strut
\end{minipage} & \begin{minipage}[t]{0.37\columnwidth}\raggedright
During the past 30 days, did anyone in your household have to: \emph{Beg} because there was not enough resources (food, cash, else) to meet essential needs (e.g.~adequate shelter, education services, health services, food, etc)?\strut
\end{minipage} & \begin{minipage}[t]{0.43\columnwidth}\raggedright
1) No, because I did not face a shortage of food 2) No; because I already sold those assets or did this activity in the last 12 months and cannot continue to do it 3) Yes 4) Not applicable\strut
\end{minipage}\tabularnewline
\begin{minipage}[t]{0.11\columnwidth}\raggedright
LhCSIEmergency2\_EN\_why\strut
\end{minipage} & \begin{minipage}[t]{0.37\columnwidth}\raggedright
What was the MAIN REASON you or other members in your household adopted these coping strategies to access essential needs?\strut
\end{minipage} & \begin{minipage}[t]{0.43\columnwidth}\raggedright
1)To access food 2)To access education services/commodities (e.g.~uniforms, books) 3)To access health services/medicines 4)To buy and repair cloths or shoes 5)To access adequate shelter 6)To access water/sanitation facilities 7)To access essential dwelling services (electricity, energy, waste disposal\ldots) 8)To pay for existing debts 9)Other\strut
\end{minipage}\tabularnewline
\begin{minipage}[t]{0.11\columnwidth}\raggedright
LhCSIEmergency3\_EN\strut
\end{minipage} & \begin{minipage}[t]{0.37\columnwidth}\raggedright
During the past 30 days, did anyone in your household have to: \emph{Sell last female animals} because there was not enough resources (food, cash, else) to meet essential needs (e.g.~adequate shelter, education services, health services, food, etc)?\strut
\end{minipage} & \begin{minipage}[t]{0.43\columnwidth}\raggedright
1) No, because I did not face a shortage of food 2) No; because I already sold those assets or did this activity in the last 12 months and cannot continue to do it 3) Yes 4) Not applicable\strut
\end{minipage}\tabularnewline
\begin{minipage}[t]{0.11\columnwidth}\raggedright
LhCSIEmergency3\_EN\_why\strut
\end{minipage} & \begin{minipage}[t]{0.37\columnwidth}\raggedright
What was the MAIN REASON you or other members in your household adopted these coping strategies to access essential needs?\strut
\end{minipage} & \begin{minipage}[t]{0.43\columnwidth}\raggedright
1)To access food 2)To access education services/commodities (e.g.~uniforms, books) 3)To access health services/medicines 4)To buy and repair cloths or shoes 5)To access adequate shelter 6)To access water/sanitation facilities 7)To access essential dwelling services (electricity, energy, waste disposal\ldots) 8)To pay for existing debts 9)Other\strut
\end{minipage}\tabularnewline
\bottomrule
\end{longtable}

\hypertarget{standardized-module---essential-needs-assessment-ena-1}{%
\section{Standardized Module - Essential Needs Assessment (ENA)}\label{standardized-module---essential-needs-assessment-ena-1}}

\hypertarget{paper-version-of-questionnaire-4}{%
\subsection{Paper Version of Questionnaire}\label{paper-version-of-questionnaire-4}}

Here is the standardized module in word format:
\href{https://github.com/WFP-VAM/RBD_FS_CH_guide_EN/blob/master/questionnaires/RBDstandardized_questionnaireLHCS_ENA.docx}{RBDstandardized\_questionnaireLHCS\_ENA}

\hypertarget{electronic-version-of-questionnaire-4}{%
\subsection{Electronic Version of Questionnaire}\label{electronic-version-of-questionnaire-4}}

Here is the standardized module in xlsform:

\href{https://github.com/WFP-VAM/RBD_FS_CH_guide_EN/blob/master/questionnaires/RBDstandardized_questionnaireLHCS_ENA.xlsx}{RBDstandardized\_questionnaireLHCS\_ENA}

\hypertarget{analysis-1}{%
\subsection{Analysis}\label{analysis-1}}

\hypertarget{example-data-set-4}{%
\subsubsection{Example data set}\label{example-data-set-4}}

Here is the example data set:

not available yet

\hypertarget{spss-syntax-4}{%
\subsubsection{SPSS Syntax}\label{spss-syntax-4}}

Here is the SPSS syntax file:

still need to do

\hypertarget{r-syntax-4}{%
\subsubsection{R Syntax}\label{r-syntax-4}}

\begin{Shaded}
\begin{Highlighting}[]
\KeywordTok{library}\NormalTok{(haven)}
\KeywordTok{library}\NormalTok{(labelled)}
\KeywordTok{library}\NormalTok{(tidyverse)}

\CommentTok{#import dataset}
\NormalTok{dataLHCS_ENAEng <-}\StringTok{ }\KeywordTok{read_sav}\NormalTok{(}\StringTok{"example_datasets/dataLHCS_ENAEng.sav"}\NormalTok{)}

\CommentTok{#Calculate HHS }
\NormalTok{dataLHCS_ENAEng<-}\StringTok{ }\KeywordTok{to_factor}\NormalTok{(dataLHCS_ENAEng)}

\CommentTok{#create a variable to specify if the household used any of the strategies by severity }
\CommentTok{#stress}
\NormalTok{dataLHCS_ENAEng <-}\StringTok{ }\NormalTok{dataLHCS_ENAEng }\OperatorTok\StringTok{ }\KeywordTok{mutate}\NormalTok{(}\DataTypeTok{stress_coping =} \KeywordTok{case_when}\NormalTok{(}
\NormalTok{LhCSIStress1_EN }\OperatorTok\StringTok{ }\KeywordTok{c}\NormalTok{(}\StringTok{"Yes"}\NormalTok{,}\StringTok{"No; because I already sold those assets or did this activity in the last 12 months and cannot continue to do it"}\NormalTok{) }\OperatorTok{&}\StringTok{ }
\NormalTok{LhCSIStress1_EN_why }\OperatorTok{==}\StringTok{ "To buy food"} \OperatorTok{~}\StringTok{ "usedstress"}\NormalTok{,}
\NormalTok{LhCSIStress2_EN }\OperatorTok\StringTok{ }\KeywordTok{c}\NormalTok{(}\StringTok{"Yes"}\NormalTok{,}\StringTok{"No; because I already sold those assets or did this activity in the last 12 months and cannot continue to do it"}\NormalTok{) }\OperatorTok{&}
\NormalTok{LhCSIStress2_EN_why }\OperatorTok{==}\StringTok{ "To buy food"} \OperatorTok{~}\StringTok{ "usedstress"}\NormalTok{,}
\NormalTok{LhCSIStress3_EN }\OperatorTok\StringTok{ }\KeywordTok{c}\NormalTok{(}\StringTok{"Yes"}\NormalTok{,}\StringTok{"No; because I already sold those assets or did this activity in the last 12 months and cannot continue to do it"}\NormalTok{) }\OperatorTok{&}
\NormalTok{LhCSIStress3_EN_why }\OperatorTok{==}\StringTok{ "To buy food"} \OperatorTok{~}\StringTok{ "usedstress"}\NormalTok{,}
\NormalTok{LhCSIStress4_EN }\OperatorTok\StringTok{ }\KeywordTok{c}\NormalTok{(}\StringTok{"Yes"}\NormalTok{,}\StringTok{"No; because I already sold those assets or did this activity in the last 12 months and cannot continue to do it"}\NormalTok{) }\OperatorTok{&}
\NormalTok{LhCSIStress4_EN_why }\OperatorTok{==}\StringTok{ "To buy food"} \OperatorTok{~}\StringTok{ "usedstress"}\NormalTok{,}
\OtherTok{TRUE} \OperatorTok{~}\StringTok{ "notusedstress"}\NormalTok{))}
\CommentTok{#Crisis}
\NormalTok{dataLHCS_ENAEng <-}\StringTok{ }\NormalTok{dataLHCS_ENAEng }\OperatorTok\StringTok{ }\KeywordTok{mutate}\NormalTok{(}\DataTypeTok{crisis_coping =} \KeywordTok{case_when}\NormalTok{(}
\NormalTok{LhCSICrisis1_EN }\OperatorTok\StringTok{ }\KeywordTok{c}\NormalTok{(}\StringTok{"Yes"}\NormalTok{,}\StringTok{"No; because I already sold those assets or did this activity in the last 12 months and cannot continue to do it"}\NormalTok{) }\OperatorTok{&}\StringTok{ }
\NormalTok{LhCSICrisis1_EN_why }\OperatorTok{==}\StringTok{ "To buy food"} \OperatorTok{~}\StringTok{ "usedcrisis"}\NormalTok{,}
\NormalTok{LhCSICrisis2_EN }\OperatorTok\StringTok{ }\KeywordTok{c}\NormalTok{(}\StringTok{"Yes"}\NormalTok{,}\StringTok{"No; because I already sold those assets or did this activity in the last 12 months and cannot continue to do it"}\NormalTok{) }\OperatorTok{&}\StringTok{ }
\NormalTok{LhCSICrisis2_EN_why }\OperatorTok{==}\StringTok{ "To buy food"} \OperatorTok{~}\StringTok{ "usedcrisis"}\NormalTok{,}
\NormalTok{LhCSICrisis3_EN }\OperatorTok\StringTok{ }\KeywordTok{c}\NormalTok{(}\StringTok{"Yes"}\NormalTok{,}\StringTok{"No; because I already sold those assets or did this activity in the last 12 months and cannot continue to do it"}\NormalTok{) }\OperatorTok{&}
\NormalTok{LhCSICrisis3_EN_why }\OperatorTok{==}\StringTok{ "To buy food"} \OperatorTok{~}\StringTok{ "usedcrisis"}\NormalTok{,}
\OtherTok{TRUE} \OperatorTok{~}\StringTok{ "notusedcrisis"}\NormalTok{))}
\CommentTok{#Emergency}
\NormalTok{dataLHCS_ENAEng <-}\StringTok{ }\NormalTok{dataLHCS_ENAEng }\OperatorTok\StringTok{ }\KeywordTok{mutate}\NormalTok{(}\DataTypeTok{emergency_coping =} \KeywordTok{case_when}\NormalTok{(}
\NormalTok{LhCSIEmergency1_EN }\OperatorTok\StringTok{ }\KeywordTok{c}\NormalTok{(}\StringTok{"Yes"}\NormalTok{,}\StringTok{"No; because I already sold those assets or did this activity in the last 12 months and cannot continue to do it"}\NormalTok{) }\OperatorTok{&}\StringTok{ }
\NormalTok{LhCSIEmergency1_EN_why }\OperatorTok{==}\StringTok{ "To buy food"} \OperatorTok{~}\StringTok{ "usedemergency"}\NormalTok{,}
\NormalTok{LhCSIEmergency2_EN }\OperatorTok\StringTok{ }\KeywordTok{c}\NormalTok{(}\StringTok{"Yes"}\NormalTok{,}\StringTok{"No; because I already sold those assets or did this activity in the last 12 months and cannot continue to do it"}\NormalTok{) }\OperatorTok{&}\StringTok{ }
\NormalTok{LhCSIEmergency2_EN_why }\OperatorTok{==}\StringTok{ "To buy food"} \OperatorTok{~}\StringTok{ "usedemergency"}\NormalTok{,}
\NormalTok{LhCSIEmergency3_EN }\OperatorTok\StringTok{ }\KeywordTok{c}\NormalTok{(}\StringTok{"Yes"}\NormalTok{,}\StringTok{"No; because I already sold those assets or did this activity in the last 12 months and cannot continue to do it"}\NormalTok{) }\OperatorTok{&}\StringTok{ }
\NormalTok{LhCSIEmergency3_EN_why }\OperatorTok{==}\StringTok{ "To buy food"} \OperatorTok{~}\StringTok{ "usedemergency"}\NormalTok{,}
\OtherTok{TRUE} \OperatorTok{~}\StringTok{ "noutusedemergency"}\NormalTok{))}

\CommentTok{#calculate Max_coping_behaviour}
\NormalTok{dataLHCS_ENAEng <-}\StringTok{ }\NormalTok{dataLHCS_ENAEng }\OperatorTok\StringTok{ }\KeywordTok{mutate}\NormalTok{(}\DataTypeTok{LhCSICat_ENA =} \KeywordTok{case_when}\NormalTok{(}
\NormalTok{  emergency_coping }\OperatorTok{==}\StringTok{ "usedemergency"} \OperatorTok{~}\StringTok{ "EmergencyStrategies"}\NormalTok{,}
\NormalTok{  crisis_coping }\OperatorTok{==}\StringTok{ "usedcrisis"} \OperatorTok{~}\StringTok{ "CrisisStrategies"}\NormalTok{,}
\NormalTok{  stress_coping }\OperatorTok{==}\StringTok{ "usedstress"} \OperatorTok{~}\StringTok{ "StressStrategies"}\NormalTok{,}
  \OtherTok{TRUE} \OperatorTok{~}\StringTok{ "NoStrategies"}\NormalTok{))}
\KeywordTok{var_label}\NormalTok{(dataLHCS_ENAEng}\OperatorTok{$}\NormalTok{LhCSICat_ENA) <-}\StringTok{ "Livelihood Coping Strategy categories - ENA version"}

\CommentTok{#Generate table of proportion of households in LhCHS  phases by Adm1 and Adm2 using weights}
\CommentTok{#Livelihood Coping Strategies }
\NormalTok{LhHCSCat_table_wide <-}\StringTok{ }\NormalTok{dataLHCS_ENAEng }\OperatorTok\StringTok{ }
\StringTok{  }\KeywordTok{drop_na}\NormalTok{(LhCSICat_ENA) }\OperatorTok
\StringTok{  }\KeywordTok{group_by}\NormalTok{(ADMIN1Name, ADMIN2Name) }\OperatorTok
\StringTok{  }\KeywordTok{count}\NormalTok{(LhCSICat_ENA , }\DataTypeTok{wt =}\NormalTok{ WeightHH) }\OperatorTok
\StringTok{  }\KeywordTok{mutate}\NormalTok{(}\DataTypeTok{perc =} \DecValTok{100} \OperatorTok{*}\StringTok{ }\NormalTok{n }\OperatorTok{/}\StringTok{ }\KeywordTok{sum}\NormalTok{(n)) }\OperatorTok
\StringTok{  }\KeywordTok{ungroup}\NormalTok{() }\OperatorTok\StringTok{ }\KeywordTok{select}\NormalTok{(}\OperatorTok{-}\NormalTok{n) }\OperatorTok
\StringTok{  }\KeywordTok{spread}\NormalTok{(}\DataTypeTok{key =}\NormalTok{ LhCSICat_ENA, }\DataTypeTok{value =}\NormalTok{ perc) }\OperatorTok\StringTok{ }\KeywordTok{replace}\NormalTok{(., }\KeywordTok{is.na}\NormalTok{(.), }\DecValTok{0}\NormalTok{) }\OperatorTok\StringTok{ }\KeywordTok{mutate_if}\NormalTok{(is.numeric, round, }\DecValTok{1}\NormalTok{)}
\end{Highlighting}
\end{Shaded}

Here is the R syntax file:

not available yet

\hypertarget{data-quality-monitoring-report}{%
\chapter{Data Quality Monitoring Report}\label{data-quality-monitoring-report}}

coming soon

\hypertarget{calculating-ch-indicators}{%
\chapter{Calculating CH Indicators}\label{calculating-ch-indicators}}

  \bibliography{book.bib,packages.bib}

\end{document}
